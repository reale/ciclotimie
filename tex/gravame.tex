\chapter*{A gravame}

\begin{verse}
    \begin{otherlanguage}{french}
        Ço dit Ysolt : Jol sai pur veir.\\
        Sachez que le sigle est tut neir.
    \end{otherlanguage}
\end{verse}

\poemtitle{i}

\begin{verse}
    alle luci dell'alba, alla strada già chiara\\
    alle torce (inutili) tra le mani\\
    al voltare della cantonata\\
    con la via che slarga verso il mare\\
    al troppo\\
    lezzo di primavera che comincia\\
    (finalmente)\\
    e ai tuoi seni barbari, per ultimi
\end{verse}

\clearpage

\poemtitle{ii}

\begin{verse}
    il confine tra veglia chiacchierina\\
    e sonno inconsapevole\\
    l’ho superato stanotte mille volte\\
    come un fiume basso da passare a guado
    in un senso prima e poi nell’altro
\end{verse}

\begin{verse}
    se ai miei occhi ogni frontiera\\
    non è mai netta indiscutibile\\
    ma frastagliata e gonfia e terra ampia\\
    è alle notti agitate e alla contesa aperta\\
    di sonno e veglia che ne son debitore
\end{verse}

\clearpage

\poemtitle{iii}

\begin{verse}
    ancora mi seduce la sostanza oscura\\
    e calda delle notti sicché corro\\
    a rifugiarmi tra le coltri\\
    preso da voglia di tana
\end{verse}

\begin{verse}
    e che siano serrate a doppia mandata\\
    le porte\\
    che non vi s'insinui\\
    agonia viscida d'inverno
\end{verse}

\begin{verse}
    e non è più il tempo che mi lascio\\
    strappare al sonno\\
    da voci trascinate nel buio\\
    ma che non lo vorrei ancora una volta\\
    non chiedetemi giuramento
\end{verse}

\clearpage

\poemtitle{iv}

\begin{verse}
    cari muri d’edera e vecchi lampioni\\
    quante volte mi foste quinta\\
    a un vagare interminato\\
    compagni nell’imbarazzo del mattino\\
    scorta di strada senza fine
\end{verse}

\begin{verse}
    ma non lasciate amici\\
    che io sia solo stanotte\\
    negli angoli gremiti di rottami\\
    si appiatta il corteo dei giorni\\
    l’affanno mi toglie le forze\\
    il non saper dire\\
    ancora\\
    sono nascosti\\
    laggiù
\end{verse}

\clearpage

\poemtitle{v}

\begin{verse}
    non è ancora allentata\\
    la morsa chiusa dell'inverno\\
    ma lo sarà presto
\end{verse}

\begin{verse}
    come un esercito schierato\\
    come le schiere rosse dell'armata fuori dai reticolati\\
    così è la primavera tutt'intorno alle mura\\
    e manda lontano i canti di guerra e le grida
\end{verse}

\begin{verse}
    io non sbircio dalle finestre\\
    ma spingo un braccio oltre gli interstizi della tana
\end{verse}

\clearpage

\poemtitle{vi}

\begin{verse}
    nient’altro che\\
    un contorcersi di nero nel nero\\
    però un contorcersi regale\\
    e artificiali lucciole che non si contavano\\
    e la solita notte incorente\\
    la solita notte in problematico equilibrio\\
    la solita notte che non dà pace
\end{verse}

\clearpage

\poemtitle{vii}

\begin{verse}
    si strappa il velo del cielo\\
    si strappa ad oriente la tregua della notte\\
    troppo tardi per negarsi alla crudeltà primaverile\\
    troppo tardi per aggrapparsi all'inverno che dilegua\\
    e il giorno dilaga tra le pietre\\
    e per le vie di giovinezza antica\\
    le membra stracche dal carico inutile e lungo\\
    si riposano\\
    e non si risparmiano gli schiaffi sulle spalle dei compagni\\
    che ormai la fatica è compiuta\\
    e poi a stare tutti insieme si sente meno\\
    questo giorno che comincia stupito e senza ombre\\
    fiaccole opache e inutili\\
    estinte a una a una\\
    con gesto di carnefice
\end{verse}

\clearpage

\poemtitle{viii}

\begin{verse}
    giorno pieno
\end{verse}

\begin{verse}
    per la finestra spalancata\\
    si versa dentro alla stanza\\
    il cielo di nuvolaglia
\end{verse}

\begin{verse}
    un brandello di sonno ancora\\
    strappato agli uffici mattinali
\end{verse}

\begin{verse}
    è il sogno del padre\\
    finalmente remoto\\
    (scoppio d’odio o di passione tardiva?)\\
    l’estrema difesa
\end{verse}

\begin{verse}
    risvegliarsi poi alla ferraglia\\
    del tram\\
    che riempie la strada
\end{verse}

\clearpage

\poemtitle{ix}

\begin{verse}
    non è ancora\\
    o non è più il tempo\\
    di prestare orecchio\\
    a un remoto dolore carnale\\
    eppure\\
    troppo di me c'è rimasto incastrato\\
    onde il continuo riandare
\end{verse}

\clearpage

\poemtitle{x}

\begin{verse}
    sgorgare di voci\\
    rorida messe\\
    di spighe feconde
\end{verse}

\begin{verse}
    sensi tesi\\
    incerta sapienza\\
    disperazione carnale
\end{verse}

\begin{verse}
    memorie di una\\
    perduta pubertà\\
    passione voluta gridare
\end{verse}

\begin{verse}
    ma soffocata nella carne\\
    un passo prima di primavera
\end{verse}

\clearpage

\poemtitle{xi}

\begin{verse}
    vessillo esposto e spregiato\\
    panno mondato di macchie\\
    lenzuolo tradito di nozze
\end{verse}

\begin{verse}
    schiocchi appiccicosi\\
    come sofferenza dolce\\
    vessillo teso nell'aria\\
    tesa essa pure di freddo\\
    che va via
\end{verse}

\begin{verse}
    sera lontana di primavera\\
    lontana sera di strazio\\
    di vento e di sensi\\
    di dolorosa nascita\\
    di remota venuta al mondo\\
    scaglia di notte più densa\\
    straccio tessuto troppo in fretta\\
    telo riempito di paure\\
    cortina spregiata
\end{verse}

\begin{verse}
    messe di vento\\
    facies di antiche nozze\\
    forma di geometria non dedotta\\
    incalzare non placato\\
    non ancora
\end{verse}

\clearpage

\poemtitle{xii}

\begin{verse}
    c’è dentro più di durezza\\
    negli artigli d'astore dell'inverno\\
    aggrappati alle carni\\
    o nel dilatarsi del buio\\
    di questa notte?
\end{verse}

\begin{verse}
    s'immagina lontana l'alba\\
    ma non è\\
    un gallo riempie la cavità della notte\\
    di un gracchiare fuori tempo\\
    forse per la troppo incongrua\\
    dilatazione
\end{verse}

\begin{verse}
    è fresco
\end{verse}

\begin{verse}
    il lenzuolo un cencio lasciato in un angolo\\
    il gallo neppure fa più paura\\
    sta lì muto o stride ogni tanto
\end{verse}
