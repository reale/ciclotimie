\begin{volumetitlepage}
	\volumetitle{Inverni}
	\volumeheader{Inverni}
	\bigskip\bigskip\bigskip
	\volumeepigraph{
		\begin{verse}
			\begin{otherlanguage}{greek}
				τὸ τέλος ὁ χρόνος ἀπαιτεῖ
			\end{otherlanguage}
		\end{verse}
	}
	\volumeattribution{Epitaffio di Sicilo}
\end{volumetitlepage}

\poemtitle{i}

	\begin{verse}
		virgulto tenero d’autunno\\
		novembre nato in fretta\\
		non più che un presentimento\\
		di braci e castagne
	\end{verse}

	\begin{verse}
		la forza adorata dell’estate\\
		corse via in fretta\\
		ora dita gelide tentano gli scuri
	\end{verse}

\clearpage

\poemtitle{ii}

	\begin{verse}
		solo gesto di ribellione\\
		offrirsi alle raffiche nudi\\
		respirare aria secca tagliente\\
		lasciarsi scovare qualcosa\\
		da qualche parte
	\end{verse}

\clearpage

\poemtitle{iii}

\begin{artItem}
	Antonín Slavíček, \begin{otherlanguage}{czech}%
		Rybník%
	\end{otherlanguage}
\end{artItem}

	\begin{verse}
		questo novembre\\
		ci è scappato tra le dita\\
		come un cucciolo irrequieto
	\end{verse}

\clearpage

\poemtitle{iv}

	\begin{verse}
		la stagione già inclina\\
		al tempo che è più matura\\
		più piena la sua luce\\
		ma c’è rimasto l’inverno\\
		impigliato tra i gesti
	\end{verse}

\clearpage

\poemtitle{v}

	\begin{verse}
		che la notte di gennaio\\
		cuore di nebbia e strida di civette\\
		non sia inospite a pensieri meridiani\\
		— che sono disumani\\
		— come sappiamo
	\end{verse}
