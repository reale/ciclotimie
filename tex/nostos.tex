\chapter*{Nostos}

\poemtitle{notturno}

\begin{verse}
    quando ti sogno\\
    mia capitale\\
    sempre ti so assediata di sonno\\
    strapiena di strade\\
    disfatta
\end{verse}

\begin{verse}
    (mia te lo dico\\
    ad colorandam possessionem\\
    per lo struggimento delle tue piazze\\
    precipue de nocte\\
    quod oculis nocet)
\end{verse}

\begin{verse}
    formicolante d'ombre\\
    amiche a chi sa bucarle\\
    dimora a randagi come allumano i lampioni\\
    colante dal tuo centro\\
    pigiatura matura d'uve nell'imbuto
\end{verse}

\clearpage

\poemtitle{vicoli}

\begin{verse}
    i vicoli all'alba\\
    sono campi seminati\\
    di rebus accartocciati\\
    di scaglie di luce\\
    di leccornie esanimi\\
    arengo di gabbiani
\end{verse}

\clearpage

\poemtitle{laurentino 38}

\begin{verse}
    sei confine, grumo rappreso di cemento\\
    sull'orlo dell'agro, intrico di gusci
\end{verse}

\begin{verse}
    sei gioco di forme, finezza sospesa\\
    di macchina
\end{verse}

\begin{verse}
    sei sogno di logos orgoglioso
\end{verse}

\begin{verse}
    sei ingenuità, sei indugio, sei concrezione
    lenta a crescere
\end{verse}

\begin{verse}
    sei le pastoie del burocrate, e le ragioni\\
    troppo brevi della politica
\end{verse}

\begin{verse}
    sei colpa
\end{verse}

\begin{verse}
    sei vita non voluta, vita di gramigna\\
    che ti fa suo
\end{verse}

\begin{verse}
    sei enclave, ricetto di guai, terra mala
\end{verse}

\begin{verse}
    sei coagulo di desiderio
\end{verse}

\begin{verse}
    sei tu (dicono) a guardare di laggiù\\
    l'E42\\
    con sguardo da escluso\\
    bramoso\\
    di schiavo
\end{verse}

\clearpage

\poemtitle{treni a portonaccio}

\begin{verse}
    chi non sa gli arrivi e le partenze\\
    e le manovre e le macchine fumanti\\
    e i lamenti da gabbiano dei freni\\
    e l'ordine saldo e onesto del capo\\
    chi non sa tutto questo\\
    non ti conosce, Roma
\end{verse}

\begin{verse}
    e lasciamogli pure quelle pietre\\
    consunte dal gregario\\
    a chi di te non guarda\\
    che i fasti antichi
\end{verse}

\begin{verse}
    noi amiamo il vapore nelle notti d'inverno\\
    e la fatica uguale di macchine e uomo\\
    la dignità tua nuova\\
    la tua vita di metallica fenice
\end{verse}

\clearpage

\poemtitle{autunnale per san lorenzo}

\begin{verse}
    chi sa la tua controra di quartiere\\
    romana tanto poco e il tuo\\
    antimediterraneo pan\\
    che gioca a rimpiattino\\
    per strade ferrate e fabriche\\
    vòlte a settentrione?
\end{verse}

\begin{verse}
    chi sa la nostalgia d'un pomeriggio\\
    lieve di domenica d'autunno\\
    che sono spente le tède\\
    —ma appena appena—\\
    delle tue ebdomadarie tesmoforie\\
    e già si danno per sedotte\\
    a un raziocinio ossidato e caldo\\
    le tue strade?
\end{verse}

\begin{verse}
    chi sa di quanto tu infittisci\\
    la trama della mia appartenenza\\
    sarcitura di opera e di umori?
\end{verse}

\clearpage

\poemtitle{casilino 23}

\begin{verse}
    la nebbia—quel che resta\\
    alla terra di agro scomparso—\\
    si lascia cucire da fili\\
    di lampioni senza ombrarsi
\end{verse}

\begin{verse}
    io mi ci rannicchio appena\\
    a un passo anche stanotte\\
    dalle falangi enormi\\
    sul ciglio della messe\\
    di gioie domestiche\\
    a decine di migliaia
\end{verse}

\begin{verse}
    e voglio che il mio luogo\\
    —questo tra i miei luoghi—\\
    trattenga la memoria\\
    di uno sguardo aggrottato\\
    di una rinuncia feroce\\
    di una definitiva non-scelta
\end{verse}

\clearpage

\poemtitle{hortus conclusus}

\begin{verse}
    che la mia casa sia\\
    non impedimento di muri\\
    non schermo al quartiere\\
    non spazio sottratto\\
    e nemmeno\\
    tana\\
    o torrione\\
    o faro\\
    o nascondiglio
\end{verse}

\begin{verse}
    voglio muri fatti di cotone\\
    di carta\\
    di zucchero filato\\
    di ragnatele\\
    o muri per niente
\end{verse}

\begin{verse}
    che la mia casa sia\\
    non filtro al respiro\\
    all'odore di fuori\\
    e nemmeno\\
    nasconda nel ventre\\
    vergognoso e sporco\\
    l'amore\\
    ma lo gridi ai passanti
\end{verse}

\begin{verse}
    nella notte del quartiere\\
    stanato da voglia di strade\\
    voglio sentirmi a casa\\
    più che dietro le tende\\
    della mia casa
\end{verse}

\clearpage

\poemtitle{notti romane}

\begin{verse}
    notti di queste estate urbanizzata\\
    questa è la volta che vengo a stanarvi\\
    sui viali infilzati da fari stizziti\\
    corso Trieste l'Ostiense via Ugo Ojetti\\
    o meglio ancora sui ponti fuori mano\\
    sulle campate che scavalcano i binari\\
    cioè quel che fa la Serenissima al Collatino\\
    o meglio ancora agli angoli introversi\\
    lungo gli archi slabbrati del Raccordo\\
    o nell'ombra che fanno le fronde\\
    uscendo ribelli come i ricci d'un moretto\\
    come ad esempio al Mandrione e al Casaletto\\
    o lungo i recinti scuri delle ville urbane\\
    strette da grate inutili e invitanti\\
    e se mi scappa l'uzzolo\\
    tra i marmi e i travertini dei portoni\\
    che stanno intorno alle piazze sdegnose\\
    come la piazza Euclide o piazza\\
    di Buenos Aires
\end{verse}

\begin{verse}
    notti d'estate romana\\
    in fondo in fondo il dubbio mi resta\\
    se riesco a conoscervi come si deve\\
    ma si fa del nostro meglio\\
    e poi se non trovo voi\\
    non resto di sicuro a bocca asciutta\\
    nell'aria di fiori marciti e piombo caldo\\
    araldi prosseneti paraninfi?\\
    ne trovo quanti voglio\\
    chi mi vieta di amare in loro vostra specie?
\end{verse}
