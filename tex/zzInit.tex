%% ----------------------------------------------------------------------------
%% John's LaTex package and environment setup - Not document type specific]
%% V1.8 - 11 Jan 2017
%% ----------------------------------------------------------------------------
% John Fogarty
% Web:    http://www.jfogarty.org
% E-Mail: johnhenryfogarty@gmail.com
% --------------------------------------------------------------------------
% This work may be distributed and/or modified under the conditions of the
% LaTeX Project Public License, either version 1.3 of this license or
% (at your option) any later version.
% The latest version of this license is in
%      http://www.latex-project.org/lppl.txt
% and version 1.3 or later is part of all distributions of 
%      LaTeX version 2005/12/01 or later.
% --------------------------------------------------------------------------
\usepackage{etoolbox}                   % Extended macro management tools
\usepackage{xparse}                     % Extended newcommands
\usepackage{makeidx}          % Enable Index creation
\usepackage[tight]{shorttoc}            % Allow an abbreviated table of contents
\usepackage{tocloft}                    % Control TOC and create new lists

\usepackage{verse}

% Note BIBLATEX *cannot* be enabled with these packages.
%\usepackage[backend=biber]{biblatex}   % Use the BIBER citation generator
%\usepackage[backref=true]{biblatex}    % Bibliography with back hyperlink.

\ifthenelse{\boolean{NOAUTHORS}}{}{
  \usepackage[biblabels]{authorindex}   % Enable Authors Cite list creation
}
\usepackage{multicol}                   % Multiple columns for the index
\usepackage{blindtext}                  % Nonsense text fillers
\usepackage{ifxetex,ifluatex}           % processor conditionals
\usepackage{wrapfig}                    % Support wrapfigure environment
\usepackage{endnotes}                   % Support endnotes.
\usepackage{titling}                    % fancy titles with prefix and suffix

\usepackage[T1]{fontenc}        % enables special graphics
\usepackage{textcomp}           % LaTex fonts and glyph encodings such as \textonehalf
\usepackage{lmodern}            % LaTex modern math
\usepackage[utf8]{inputenc}             % allows limited support for non-ascii UTF8 codes in text
\usepackage{graphicx}                   % graphics driver support (math/images for PDFs)

\usepackage[english]{babel}             % Support for non-english typesetting
\usepackage[autostyle, english = american]{csquotes} % Backquotes
% XXX %\usepackage{titlesec}                % title, chapter and section formatting
\usepackage{enumerate}          % styles enumerated lists

\usepackage{changepage}           % page and margin adjustments

% Note ~/.config/texstudio/subcaption_jf.cwl file needed for autocompletion in TexStudio.
\ifthenelse{\boolean{SUBCAPTIONS}}{
  \usepackage{subcaption}               % subcaptions in figures]
}{}

% Math and Software 
%\usepackage{amsmath}             % American math (cannot use with wasysym)
\usepackage{mathtools}
\usepackage{wasysym}              % Additional symbols
\usepackage{amssymb}              % American math symbols
\usepackage{latexsym}             % LaTex symbols
\usepackage{amsfonts} % experimental
\usepackage{bm}       % math bold - experimental

%\usepackage{fontspec}    % -- Use with XeLaTex only for OpenType fonts
%\usepackage{lilyglyphs}  % -- Use with XeLaTex only, Musical notations

%\usepackage{listings}          % Computer language source code typesetting
%\usepackage{tikz}            % 3D plots - Defined in Parameters.
%\usepackage{pgfplots}                  % Complex graphs, charts, and plots

%\usepackage[nottoc]{tocbibind}         % Place the References, Index in TOC
%\usepackage[makeindex]{imakeidx}        % Make index hyperref to body
% Urls and Hrefs that work in PDFs [should be last] package.
\usepackage[hyphens]{url}  

\ifthenelse{\boolean{NOHYPERREF}}{
  \newcommand\hypersetup[1]{}%
  \newcommand{\href}[2]{Href=#1}%
  \newcommand{\nameref}[1]{NameRef=#1}%
}{%
  \ifthenelse{\boolean{TOHTML}}{
    \usepackage[plainpages=false]{hyperref}
  }{
    \usepackage[pdftex,
            pdfauthor={\TheAuthor},
            pdftitle={\TheMainTitle: \TheSubTitle},
            pdfsubject={\TheSubjectArea},
            pdfkeywords={\TheKeywords},
            plainpages=false]{hyperref}
  }
}

\usepackage{xstring}                    % String manipulation macros
\usepackage{stringstrings}              % String manipulation macros
\usepackage{xspace}                     % Useful in parameterless macros

%--------------------------------------------------------------------
% controls the TOC, Footnote, and URL/Hyperref link colors.
\colorlet{HtmlUrlColor}{green!22!black}% The color for named HREFs/URLs
\colorlet{HtmlCiteColor}{green!50!black}% The color for citations
\colorlet{HtmlLinkColor}{blue!70!black}% The color for in text links

\ifthenelse{\boolean{TOPRESS}}{
  \hypersetup{colorlinks=true, citecolor=black, filecolor=black, linkcolor=black, urlcolor=black}%
}{
  \ifthenelse{\boolean{NOCOLOR}}{
    \hypersetup{colorlinks=true, citecolor=black, filecolor=black, linkcolor=black, urlcolor=black}
  }{
    \ifthenelse{\boolean{TOHTML}}{
      \hypersetup{colorlinks=true,
          citecolor=HtmlCiteColor,
          filecolor=green,
          linkcolor=HtmlLinkColor,
          urlcolor=HtmlUrlColor}
    }{ 
      \hypersetup{colorlinks=true, linkcolor=black, urlcolor=blue}
    }
  }
}

%%---------------------------------------------------------------------
\DeclareUrlCommand\Code{\urlstyle{rm}}
\expandafter\def\expandafter\UrlBreaks\expandafter{\UrlBreaks  
\do\/\do\a\do\b\do\c\do\d\do\e\do\f\do\g\do\h\do\i\do\j\do\k
\do\l\do\m\do\n\do\o\do\p\do\q\do\r\do\s\do\t\do\u\do\v
\do\w\do\x\do\y\do\z
\do\A\do\B\do\C\do\D\do\E\do\F\do\G\do\H\do\I\do\J\do\K
\do\L\do\M\do\N\do\O\do\P\do\Q\do\R\do\S\do\T\do\U\do\V
\do\W\do\X\do\Y\do\Z}

%%---------------------------------------------------------------------
%% Hyperlinks removed on PRINT edition.
\ifthenelse{\boolean{TOPRESS}}{%
  \newcommand{\HREFG}[2]{\textsf{#1} [\url{#2}]}%
  \newcommand{\HREFX}[3]{\textsf{#1} [\Code{#3}]}%
  \newcommand{\URL}[1]{[{\allowbreak{\detokenize{#1}}}]\xspace}%
  \newcommand{\NameRef}[2][] {[see:{#2}]}
  \newcommand{\AMAZON}[1]{}
  %\newcommand{\YOUTUBE}[2]{}
  %\newcommand{\GITHUB}[1]{Github}
}{%
  \ifthenelse{\boolean{RAWHTML}}{%
    \newcommand{\HREFG}[2]{\HCode{<a href="#2" >}\textsf{#1}\HCode{</a>}\xspace}%
    \newcommand{\HREFX}[3]{\HCode{<a href="#2" >}\textsf{#1}\HCode{</a>}\xspace}%
  }{%
    \newcommand{\HREFX}[3]{\textsf{\href{#2}{#1}}\xspace}
    \newcommand{\HREFG}[2]{\textsf{\href{#2}{#1}}\xspace}
  }%
  \newcommand{\URL}[1]{\url{#1}\xspace}%
  \newcommand{\NameRef}[2][]{[see:\nameref{#2}]}%
  \newcommand{\AMAZON}[1]{\HREF{[@Amazon]}{#1}}
}

%%---------------------------------------------------------------------
%% \HREF: A variable hyperlink web reference.
%% When output to a printable PDF, HREFs need to be formatted so
%% the printed page contains the URL that can be typed into a web
%% browser. In an epub, html, or online pdf, the URL doesn't need to
%% shown and just the hyperlinked human readable text is sufficient.
%% Unfortunately URLs can't always be typeset properly (they can't be
%% hyphenated, so LaTeX has difficulty figuring out how to break them
%% across lines. When this happens, you need to use the optional
%% first argument, which is the URL with line breaks embedded in it.
%% This can't be the real URL since that would break it when used in 
%% the epub/html/etc. Tedious, but thats how it is.
%%
%% \HREF{Human readable text}{URL including http:// etc.}
%% \HREF[optional Printable URL]{Human readable text}{URL}
%%
\newcommand{\HREF}[3][]{%
  \ifthenelse{\isempty{#1}}{%
    \HREFG{#2}{#3}%
  }{%
    \HREFX{#2}{#3}{#1}%
  }%
}%

\newcommand{\GITHUB}[1]{\HREF{[@GitHub: #1]}{https://github.com/#1}}
\newcommand{\YOUTUBE}[3][]{\HREF[#1]{[@YouTube: #2]}{#3}}
\newcommand{\WIKI}[3][]{\HREF[#1]{[@Wikipedia: #2]}{#3}}

%%---------------------------------------------------------------------
%% Set running headings at top of pages.
\newcommand{\RightRunningHeads}[1]{\renewcommand\rightmark{#1}}
\newcommand{\LeftRunningHeads}[1]{\renewcommand\leftmark{#1}}

\newcommand{\RunningHeads}[1]{
    \LeftRunningHeads{#1}%
    \RightRunningHeads{#1}%
}

%%---------------------------------------------------------------------
%% Index generation is optional.
\newcommand{\InitMakeIndex}{
  \ifthenelse{\boolean{ONECOLUMN}}{%
    \usepackage[columns=1]{idxlayout} % eBooks do not need a 2 column index.
  }{}%
}

\ifthenelse{\boolean{NOINDEX}}{%
  \newcommand\Index[1]{}%
  \newcommand\PrintIndex{}%
}{%
\newcommand\Index[1]{\index{#1}}%
\newcommand\PrintIndex{%
  \RunningHeads{INDEX}%
  \printindex}%
}%

%%---------------------------------------------------------------------
%% Cited Authors list generation is optional.
\newcommand{\EmitPrintAuthorIndex}{
  \ifthenelse{\boolean{ONECOLUMN}}{%
    \printauthorindex%
  }{%
    \begin{multicols}{2}%
      \printauthorindex%
    \end{multicols}%
  }%
}

\newcommand{\PrintAuthorIndex}{
  \ifthenelse{\boolean{NOAUTHORS}}{}{%
    \ifthenelse{\boolean{TOHTML}}{%
      \renewcommand\indexspace{
    }%
  }{}%
    \cleardoublepage\phantomsection%
    \RunningHeads{Authors Cited}
    \chapter*{Authors Cited}%
    \addcontentsline{toc}{chapter}{Authors Cited}%
    \EmitPrintAuthorIndex
  }%
}%

%---------------------------------------------------------------------
% Add Index[optional index text]{permuted section name}{index text}
\newcommand{\CatIndex}[2]{\Index{#1!\textsf{#2}}}

\newcommand{\Ix}[3][]{%
  \ifthenelse{\isempty{#1}}{\CatIndex{#2}{#3}\Index{#3}}{\CatIndex{#2}{#1}\Index{#1}}%
}

\newcommand{\Inx}[2][]{\ifthenelse{\isempty{#1}}{#2\Index{#2}}{#2\Index{#1}}}
\newcommand{\PInx}[3][]{#3\Ix[#1]{#2}{#3}}

%%---------------------------------------------------------------------
\newcommand{\ParagraphIndent}[1]{%
  \setlength\parindent{#1ex}%
}

\newcommand{\Indent}{%
  \ParagraphIndent{4}%
}

\newcommand{\StopIndent}{%
  \ParagraphIndent{0}%
}

%%---------------------------------------------------------------------
% Manipulate line spacing
\newcommand{\NegLineSkip}{%
  \ifthenelse{\boolean{TOHTML}}{%
    \vspace*{-0.9\baselineskip}%
  }{%
    \vspace*{-0.5\baselineskip}%
 }%
}

\newcommand{\NoParagraphSkip}{\parskip 0pt}
\newcommand{\ParagraphSkip}{\parskip 0pt}

%%---------------------------------------------------------------------
% Defines the heading text for a chapter's notes.
\newcommand{\ChapterNotesName}[1]{%
  \ifthenelse{\boolean{CHAPTERNOTES}}{%
    \renewcommand{\notesname}{Notes [#1]}%
  }{%
    \addtoendnotes{
      \bigskip\StopIndent
      \large\textsf{Chapter \thechapter\, --- #1}
      \normalsize\normalfont\bigskip
    }
  }%    
}

%%---------------------------------------------------------------------
% After each chapter reset the note numbering and output the 
% accumulated chapter notes if requested.
\newcommand{\ChapterSuffix}{%
  \ifdefined\IsOutline\else
    \ifthenelse{\boolean{CHAPTERNOTES}}{%
      \theendnotes%
      \setcounter{endnote}{0}%
    }{
      \setcounter{endnote}{0}%
    }%
  \fi%
}

%%---------------------------------------------------------------------
\newcommand{\ChapterPrefix}{%
  \ifdefined\IsOutline\else%
    \ChapterSuffix%
    \clearpage%
  \fi%
}

%%---------------------------------------------------------------------
%% Redefine the \chapter command to include book characteristics
\let\originalchapter\chapter
\RenewDocumentCommand{\chapter}{ s o m }{% s:star, o:opional, m:mandatory
   \IfBooleanTF{#1}{%
      \originalchapter*{#3}%
   }{%
      \ifdefined\IsInBody\ChapterPrefix\else\def\IsInBody{true}\fi%
      \originalchapter{#3}%
      \ChapterNotesName{#3}%
   }%
}

%%---------------------------------------------------------------------
% Outputs the Book End Notes section, if needed.
\newcommand{\BookEndNotes}{%
  \ifthenelse{\boolean{CHAPTERNOTES}}{}{%
    \cleardoublepage\phantomsection%
    \addcontentsline{toc}{chapter}{Notes by Chapter}%
    \theendnotes%
  }%
}

%%---------------------------------------------------------------------
\newcommand{\LineSeparator}{%
    \begin{center}%
        \line(1,0){250}%
    \end{center}%
}

\newcommand{\MusicNote}{\twonotes\xspace}

%%---------------------------------------------------------------------
%% Format Specific notations - Identifies oprions set in output.
%%
\newcommand{\IntroSpec}{\ifthenelse{\boolean{NOBODY}}{IntroOnly\xspace}{}}%
\newcommand{\HtmlSpec}{\ifthenelse{\boolean{TOHTML}}{HTML\xspace}{}}%
\newcommand{\MobiSpec}{\ifthenelse{\boolean{TOMOBI}}{Mobi\xspace}{}}%
\newcommand{\EpubSpec}{\ifthenelse{\boolean{TOEPUB}}{Epub\xspace}{}}%
\newcommand{\MonoSpec}{\ifthenelse{\boolean{NOCOLOR}}{Grayscale\xspace}{}}%
\newcommand{\PressSpec}{\ifthenelse{\boolean{TOPRESS}}{PRINT\xspace}{}}%

\newcommand{\DraftSpec}{\ifthenelse{\boolean{FINALFORM}}{}%
{DRAFT--\HtmlSpec\MobiSpec\EpubSpec\MonoSpec\PressSpec\IntroSpec}}%

\newcommand{\FullDraftSpec}{\ifthenelse{\boolean{FINALFORM}}{}%
{ [-----\DraftSpec---]}}%

%%---------------------------------------------------------------------
%% HTML Specific Command overrides...
\ifthenelse{\boolean{TOHTML}}{%
    \renewcommand\LineSeparator{%
        \begin{center}%    
        \ifthenelse{\boolean{FINALFORM}}{%
             ---------------------------------
        }{%
             ---------\DraftSpec---------
      }%
        \end{center}%
    }%
    \renewcommand{\MusicNote}{}%
}{}%

%%---------------------------------------------------------------------
\newcommand{\BigBold}[1]{
  \begin{flushleft}
  {\large{\textbf{--- #1 ---}}} \\
  \end{flushleft}
}

%%---------------------------------------------------------------------
% Define source directory for images as a parent directory or
% as a subdirectory.
\ifthenelse{\boolean{PARENTIMAGES}}{%
  \newcommand{\ImageSource}{../images}%
}{%
  \newcommand{\ImageSource}{images}%
}

%%---------------------------------------------------------------------
% Include a graphic image with MISSING IMAGE text if not found.
%   \IMAGE[options]{filenameWithFormat}
\newcommand{\IMAGE}[2][]{%
  \ifthenelse{\boolean{NOIMAGES}}{%
    \large\texttt{[ IMAGE:\detokenize{#2} ]}
    \normalsize
  }{%
    \ifthenelse{\boolean{FINALFORM}}{%
          \includegraphics[#1]{\ImageSource/#2}
    }{%
       \IfFileExists{\ImageSource/#2}{%
           \includegraphics[#1]{\ImageSource/#2}
      }{%
        \large{MISSING IMAGE:
        \texttt{\ImageSource/\detokenize{#2}}}\normalsize
      }%  
    }%
  }%
}

%%---------------------------------------------------------------------
% Include a graphic image that selects type and _BW version as needed.
%   \Image[options]{namebase}
\newcommand{\Image}[2][]{%
 \ifthenelse{\boolean{NOCOLOR}}{%
  \IfFileExists{\ImageSource/#2_BW.jpg}{\IMAGE[#1]{#2_BW.jpg}}{%
   \IfFileExists{\ImageSource/#2_BW.png}{\IMAGE[#1]{#2_BW.png}}{%
    \IfFileExists{\ImageSource/#2.jpg}{\IMAGE[#1]{#2.jpg}}{%
     \IMAGE[#1]{#2.png}%
    }%        
   }%       
  }%        
 }{%
  \IfFileExists{\ImageSource/#2.jpg}{\IMAGE[#1]{#2.jpg}}{%
   \IMAGE[#1]{#2.png}%
  }%        
 }%
}

%%---------------------------------------------------------------------
%%---------------------------------------------------------------------
% Define an additional list file (.los) for Figure Sources
\def\theFigureSource{x}%

\newcommand{\listsourcename}{List of Figure Sources}
\newlistof{sources}{los}{\listsourcename}

% Outputs the current figure source line to the .los.
\newcommand*{\iGenFigureSource}[1]{%
  \ifthenelse{\equal{#1}{x}}{}{%
    \addcontentsline{los}{section}{\protect\numberline{\thefigure}#1}%
  }% 
}

% Outputs the figure source line to the .los file and clears it.
\newcommand*{\iFigureSource}{{%
  \iGenFigureSource{\theFigureSource}%
  \gdef\theFigureSource{x}%
}}

%%---------------------------------------------------------------------
%%   SetFigureSource{text of figure source attribution}
%% Must be used immediately before FigureWrapped,etc. command
\newcommand*{\SetFigureSource}[1]{%
  \gdef\theFigureSource{#1}%
}

%%---------------------------------------------------------------------
%% FigureSource{text of figure source attribution}
%% Must be used INSIDE a figure environment
\newcommand*{\figureSource}[1]{{%
  \SetFigureSource{#1}%
  \iFigureSource%
}}

%%---------------------------------------------------------------------
% A column/page centered figure with caption.
%  \FigureCentered[label]{width}{caption}{imagefilenamebase}
\newcommand{\FigureCentered}[4][]{%
  \begin{figure}[ht]%
    \begin{center}%
      \ifthenelse{\isempty{#2}}%
        {\Image[width=0.75\textwidth]{#4}}%
        {\Image[width=#2\textwidth]{#4}}%
          \ifthenelse{\isempty{#3}}{}{\caption{#3}}%
          \ifthenelse{\isempty{#1}}{}{\label{fig:#1}}%
          \iFigureSource%
    \end{center}%
  \end{figure}%
}

% Koma-script caption indenting is needed for figure wrapped captions, otherwise the
% text is all squished to the right. Not supported outside Koma-script.
\ifthenelse{\isnamedefined{setcapindent}}{}{%
    \newcommand{\setcapindent}[1]{}%
}

% FigureWrapped (right justified) (wrapped with text)
%  \FigureWrapped[label]{width}{caption}{imagefilenamebase}
%  Width is a text area percentage: 0.33 is 1/3 of text width.
\newcommand{\iFigureWrapped}[4][]{
  \ifthenelse{\boolean{ONECOLUMN}}{%
    % Text wrapped figures tend to render poorly in ePubs. Its
    % better to just center the figure and be done with it.
  \FigureCentered[#1]{#2}{#3}{#4}%  
  }{%
  \begin{wrapfigure}{R}{#2\textwidth}%
        \setcapindent{1em}%
    \centering%
        \Image[width=#2\textwidth]{#4}%
    \ifthenelse{\isempty{#2}}{}{\caption{#3}}%
    \ifthenelse{\isempty{#1}}{}{\label{fig:#1}}%
        \iFigureSource%
  \end{wrapfigure}%
  }%
}%

\newcommand{\FigureWrapped}[4][]{
  \ifthenelse{\isempty{#2}}%
      {\iFigureWrapped[#1]{0.30}{#3}{#4}}%
      {\iFigureWrapped[#1]{#2}{#3}{#4}}%
}%

%%---------------------------------------------------------------------
%% \BookMath wrapper for equations -- better HTML/EPUB formatting.
%% PdfLaTeX typesetting of math works fine, but htlatex does a poor job 
%% when aligning maths and often generates bad html to boot.
%% A workaround is to format for book, display the PDF at 400% with
%% Okular and save images as PNG files for each equation.
%% Make sure to capture the entire text width (about 1730 pixels).
%% Save .png files to images/equations directory.
%% Wrap each equation with:
%%   \BookMath{imageName}{
%%      the typically multiline, often \begin{align*} wrapped math
%%    }
\newcommand{\BookMath}[2]{
  \ifthenelse{\boolean{IMAGEMATH}}{%
    \begin{center}%
        % width here doesn't seem to affect epub image size. This makes a
        % nice size for the image in the ePub PDF file. For the actual
        % epub html, the image should be selected from the textwidth.
        % Leave about 1/2 a character of whitespace top and bottom.
        \Image[width=0.80\textwidth]{equations/#1}% Requires a 400% oversized image
    \end{center}%
  }{%
    #2%
  }%
}

%%---------------------------------------------------------------------
% Convenience figure reference that adds 'figure' and fig:
\newcommand{\figref}[1]{figure \ref{fig:#1}}

%%---------------------------------------------------------------------
\newcommand{\QuoteParagraph}[2]{
  \begin{adjustwidth}{1cm}{1cm}
  --- \sffamily#1\normalfont\ifthenelse{\isempty{#2}}{}{ --- #2}
  \end{adjustwidth}
}

%%---------------------------------------------------------------------
\newcommand{\ChapterQuote}[2]{
  \begin{adjustwidth}{1cm}{1cm}
  --- \sffamily#1\normalfont\ifthenelse{\isempty{#2}}{}{%
      \NoParagraphSkip\begin{flushright}--- #2\end{flushright}
    }%
  \end{adjustwidth}
}

%%---------------------------------------------------------------------
\newcommand{\VAR}[2][]{%
  \textbf{\textsf{#2}}%
}

%%---------------------------------------------------------------------
\newenvironment{boxedarea}
    {\begin{center}
    \begin{tabular}{|p{0.9\textwidth}|}
    \hline\\
    }
    { 
    \\\\\hline
    \end{tabular} 
    \end{center}
    }

%%--------------------------------------------------------------------
%% \noparskip in paragraph to suppress blank line after this one.
%%\let\svpar\par
%%\edef\svparskip{\the\parskip}
%%\def\revertpar{\svpar\setlength\parskip{\svparskip}\let\par\svpar}
%%\def\noparskip{\leavevmode\setlength\parskip{0pt}%
%%  \def\par{\svpar\let\par\revertpar}}

%--------------------------------------------------------------------
% LaTeX Processor independent environment variable access.
% \getenv[\targetname]{envvarname}
\ifxetex
  \usepackage{catchfile}
  \newcommand\getenv[2][]{%
    \immediate\write18{kpsewhich --var-value #2 > \jobname.tmp}%
    \CatchFileDef{\temp}{\jobname.tmp}{\endlinechar=-1}%
    \if\relax\detokenize{#1}\relax\temp\else\let#1\temp\fi}
\else
  \ifluatex
    \newcommand\getenv[2][]{%
      \edef\temp{\directlua{tex.sprint(
        kpse.var_value("\luatexluaescapestring{#2}") or "" ) }}%
      \if\relax\detokenize{#1}\relax\temp\else\let#1\temp\fi}
  \else
    \usepackage{catchfile}
    \newcommand{\getenv}[2][]{%
      \CatchFileEdef{\temp}{"|kpsewhich --var-value #2"}{\endlinechar=-1}%
      \if\relax\detokenize{#1}\relax\temp\else\let#1\temp\fi}
  \fi
\fi  
% End of initialization macros  
%--------------------------------------------------------------------
