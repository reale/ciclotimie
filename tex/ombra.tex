\begin{volumetitlepage}
	\volumetitle{Ombra}
	\volumeheader{Ombra}
	\bigskip\bigskip\bigskip
	\volumeepigraph{
		\begin{otherlanguage}{romanian}
			Nu mă regăsesc în el, deşi îmi pare evidentă prezenţa isteriei mele de atunci.
		\end{otherlanguage}
	}
	\volumeattribution{Emil Cioran}
\end{volumetitlepage}

\poemtitle{i}

\begin{artItem}
	Affreschi della villa di Livia a Prima Porta
\end{artItem}

	\begin{verse}
		andai per starci (poco)\\
		ispido nelle prime mattine di gennaio\\
		ma quanto tempo per farne\\
		una dolce dorata coppa di solitudine\\
		una piccola custodia di rifugi cesellati a mano?\\
		e poi ebbi ombra\\
		di una primavera troppo lunga, forse\\
		presi una di quelle primaverili sere di agosto\\
		e me ne andai
	\end{verse}

\clearpage

\poemtitle{ii}

\begin{artItem}
	Umberto Boccioni, Quelli che restano, seconda versione
\end{artItem}

	\begin{verse}
		sul bordo d'estate mi torce i sensi\\
		più amara più tesa nell'assenza\\
		l'aria del paese dove nacqui adulto
	\end{verse}

	\begin{verse}
		ma fui lesto a privarmene\\
		consentaneo al gioco di incontri\\
		e abbandoni che io chiamo vita
	\end{verse}

\clearpage

\poemtitle{iii}

	\begin{verse}
		terra hai segreti\\
		sei dolce e petrosa\\
		e poi le cavità di mare\\
		onde non sei sprovvista\\
		quando ti vennero?
	\end{verse}

	\begin{verse}
		così ti prendi il senno\\
		dei figli e uguale di chi\\
		t’ha saputa per avventura
	\end{verse}

\clearpage

\poemtitle{iv}

	\begin{verse}
		golfi lunghi\\
		orlati di rapido asfalto\\
		sorpresi ancora dall’acqua\\
		come ventri ricolmi\\
		erano forse contrade e casali\\
		ora sono nascondimento segreto\\
		della mia terra
	\end{verse}

\clearpage

\poemtitle{v}

	\begin{verse}
		fanciullezza ombrosa\\
		fanciullezza soltanto immaginata\\
		costruita nel ricordo\\
		non irraggiungibile certo\\
		ma lasciata da parte\\
		ferocemente
	\end{verse}

	\begin{verse}
		cavità dolci e strette della mia terra\\
		dolci profonde cavità della mia terra\\
		custodite per me vi prego\\
		il segreto del vostro altrove
	\end{verse}

	\begin{verse}
		dischiudetelo per me soltanto\\
		e lasciatemi l’orgoglio\\
		della mia appartenenza\\
		che possa portarmelo\\
		in giro per il mondo
	\end{verse}

	\begin{verse}
		io sono vostro iure sanguinis\\
		e guai a chi mi dice mentitore
	\end{verse}

\clearpage

\poemtitle{vi}

\begin{artItem}
	Adolph Menzel, \begin{otherlanguage}{german}%
		Das Balkonzimmer%
	\end{otherlanguage}
\end{artItem}

	\begin{verse}
		e nidi caldi d’ombra\\
		tra le case vecchie\\
		e negli angoli delle nuove\\
		che anche loro hanno segreti
	\end{verse}

	\begin{verse}
		ci si ferma a respirare\\
		odore d’incenso\\
		profumo di letto
	\end{verse}

\clearpage

\poemtitle{vii}

\begin{artItem}
	Mario Sironi, Paesaggio urbano 1921
\end{artItem}

	\begin{verse}
		alla stazione quella prima volta\\
		stavo come chi aspetta di nascere
	\end{verse}

	\begin{verse}
		poi il cenno d’una strada ignota\\
		un mistero che sorveglia le case
	\end{verse}

	\begin{verse}
		era sera e li ricordo i fuochi vivi\\
		le foglie stropicciate e allegre
	\end{verse}

	\begin{verse}
		ed erano mio padre e mia madre\\
		quella città appena incontrata
	\end{verse}

	\begin{verse}
		licenza di vivere nuovo\\
		promessa di interstizi tra il cemento
	\end{verse}

\clearpage

\poemtitle{viii}

\begin{artItem}
	Felice Casorati, Case popolari
\end{artItem}

	\begin{verse}
		uno scaffale nell'angolo\\
		stretto d'una gelateria\\
		la scatola un po' polverosa\\
		un po' piena di meraviglie\\
		e lucente di armi smaltate
	\end{verse}

	\begin{verse}
		i piccoli lampioni appena accesi\\
		nella foschia luccicante d'autunno
	\end{verse}

	\begin{verse}
		i passanti\\
		cappelli e cappotti\\
		e occhiate ammiccanti
	\end{verse}

	\begin{verse}
		e quelle strane e lente sbuffate di vento
	\end{verse}

	\begin{verse}
		tra i vocaboli ornati da guglie in pietra sponga\\
		tra le vie cittadine strette da cemento dei '60\\
		tra le strade che tentano grembi insanguinati tra le colline
	\end{verse}

	\begin{verse}
		ma vengo a vedere\\
		la vita\\
		le vite\\
		in centro\\
		quando posso
	\end{verse}

\clearpage

\poemtitle{ix}

\begin{artItem}
	Gustave Caillebotte, \begin{otherlanguage}{french}%
		Les raboteurs de parquet%
	\end{otherlanguage}
\end{artItem}

	\begin{verse}
		una spolverata di lampioni umidi\\
		viali d'oro croccante\\
		(ma in questi giorni soltanto)\\
		e pomeriggi di miele ambrato\\
		tra i condomini
	\end{verse}

\clearpage

\poemtitle{x}

\begin{artItem}
	Romano Rui, Elemento verticale
\end{artItem}

	\begin{verse}
		vengo a concedermi\\
		—perché no?—\\
		un'andata nell'aria asprigna\\
		e dolce e voltare\\
		per gioco le pagine\\
		da accartocciare vermiglie\\
		tra i passi
	\end{verse}

	\begin{verse}
		a cercare ancora un poco di miele\\
		sull'orlo della sera\\
		a godere ai mille\\
		e mille spilli d'acciaio\\
		sulla mani e il collo\\
		a raccogliere il muso al vento\\
		ogni pezzo d'autunno
	\end{verse}

\clearpage

\poemtitle{xi}

\begin{artItem}
	Enrico Castellani, Spartito
\end{artItem}

	\begin{verse}
		il luogo lo scelsi perché disgiunto\\
		nebuloso altrove\\
		ma non sapevo\\
		avesse un oltre
	\end{verse}

	\begin{verse}
		ero andato per starci poco\\
		ispido nelle mattine di gennaio
	\end{verse}

	\begin{verse}
		poi mi bastò una strada\\
		nell'ombra di giugno\\
		il gioco segreto del sole
	\end{verse}

	\begin{verse}
		la sera soltanto riposare distesi\\
		sulla pietra calda alla stazione
	\end{verse}

	\begin{verse}
		solo dopo ebbi ombra\\
		di una troppo lunga primavera
	\end{verse}

\clearpage

\poemtitle{xii}

\begin{artItem}
	Honoré Daumier, \begin{otherlanguage}{spanish}%
		Don Quixote%
	\end{otherlanguage}
\end{artItem}

	\begin{verse}
		fu avara di brama\\
		e rada e arresa\\
		la vita lassù\\
		finché restai
	\end{verse}

	\begin{verse}
		ma fu mia sorgente\\
		non quella ingombrante\\
		carnale
	\end{verse}

	\begin{verse}
		allora ancora non sapevo\\
		partenza
	\end{verse}
