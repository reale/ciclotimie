\begin{volumetitlepage}
	\volumetitle{Pan}
	\volumeheader{Pan}
\end{volumetitlepage}

\poemtitle{i}

\begin{artItem}
	Aleksandr Michajlovič Rodčenko, \begin{otherlanguage}{russian}%
		Белый круг%
	\end{otherlanguage}
\end{artItem}

	\begin{verse}
		ma sono la stessa cosa\\
		sonno veglia\\
		vigilanza incoscienza\\
		logos panico\\
		e frontiera\\
		e il grembo incerto\\
		della terra
	\end{verse}

\clearpage

\poemtitle{ii}

\begin{artItem}
	Marcel Duchamp, \begin{otherlanguage}{french}%
		Rotorelief n\textsuperscript{o}3%
	\end{otherlanguage}
\end{artItem}

	\begin{verse}
		è specchio il dio\\
		è solitario amore\\
		è sguardo eternamente\\
		volto al fluire
	\end{verse}

\clearpage

\poemtitle{iii}

\begin{artItem}
	Philippe Decrauzat, \begin{otherlanguage}{english}%
		BSBTE (Black Should Bleed To Edge)%
	\end{otherlanguage}
\end{artItem}

	\begin{verse}
		mi vieni incontro sul limite\\
		della sera o del sonno\\
		mi mostri permeabile\\
		la frontiera\\
		giacché ti si addice\\
		stare di fronte
	\end{verse}

	\begin{verse}
		pure sono tuoi per intero\\
		terrore del meriggio\\
		strada isolata\\
		folgorazione non mediata\\
		spazio senza ragione
	\end{verse}

	\begin{verse}
		tu inflessibile\\
		lume superno\\
		generi creature\\
		pietrose\\
		che sanno sguardi\\
		di pietra
	\end{verse}

\clearpage

\poemtitle{iv}

	\begin{verse}
		a cosa assomigliarti?
	\end{verse}

	\begin{verse}
		a una bestiola dei boschi —\\
		un volpino un leprotto che guizza tra l'erba\\
		una daina giovinetta dalla tenera gola —\\
		oppure a un agnello capriccioso\\
		ignaro ancora dei morsi?
	\end{verse}

	\begin{verse}
		ma sa l'agnello la rupe\\
		e la daina il signore radioso\\
		sanno il leprotto il volpino\\
		il silenzio al meriggio e i cani\\
		e l'inutile fuga?
	\end{verse}

\clearpage

\poemtitle{v}

\begin{artItem}
	Sabin Bălașa, \begin{otherlanguage}{romanian}%
		Compoziţie marină%
	\end{otherlanguage}
\end{artItem}

	\begin{verse}
		tu inflessibile\\
		lume superno\\
		generi creature\\
		pietrose\\
		che sanno sguardi\\
		di pietra
	\end{verse}

\clearpage

\poemtitle{vi}

\begin{artItem}
	Max Ernst, \begin{otherlanguage}{german}%
		Antipoden Landschaft%
	\end{otherlanguage}
\end{artItem}

	\begin{verse}
		arcade dai molti nomi\\
		tu che rechi la luce\\
		ma godi degli antri\\
		proteggi il nostro limite\\
		e salvaci dalla pazzia\\
		la peggiore la più umana\\
		tra le nostre paure
	\end{verse}
