\begin{volumetitlepage}
	\volumetitle{Déchargements}
	\volumeheader{Déchargements}

	\bigskip\bigskip\bigskip
	\volumeepigraph{
		\begin{verse}
			\begin{otherlanguage}{greek}
				καί ῥ' ὁ μὲν ἤδη πάμπαν ἐνίπλεον ᾧ ὑπὸ μαζῷ\\
				μάργος Ἔρως λαιῆς ὑποΐσχανε χειρὸς ἀγοστόν,\\
				ὀρθὸς ἐφεστηώς
			\end{otherlanguage}
		\end{verse}
	}
	\volumeattribution{Apollonio Rodio, Arg., III, 119-21}

	\bigskip\bigskip\bigskip
	\volumeepigraph{
		\begin{verse}
			\begin{otherlanguage}{hebrew}
					גן נעול אחתי כלה גל נעול מעין חתום
			\end{otherlanguage}
		\end{verse}
	}
\end{volumetitlepage}

\poemtitle{i}

\begin{artItem}
	Willem de Kooning, \begin{otherlanguage}{english}%
		Pink Angels%
	\end{otherlanguage}
\end{artItem}

\begin{poem}
	\begin{stanza}
		sappiamo dirci per che ragione\verseline
		ci perdiamo nel negativo di danze\verseline
		attacchi parate sagacie maligne\verseline
		spiegamento di difese brutali?
	\end{stanza}

	\begin{stanza}
		sappiamo dirci per che ragione\verseline
		ci incastriamo in conversari infiniti\verseline
		parole-dardo spinte sotto pelle\verseline
		cretese intrico di assalti e ferite?
	\end{stanza}

	\begin{stanza}
		sappiamo dirci per che ragione\verseline
		non c'è più che le teste dell'idra?\verseline
		lei ancora si pasce dei resti\verseline
		della nostra morta stagione
	\end{stanza}
\end{poem}

\clearpage

\poemtitle{ii}

\begin{artItem}
	Renato Guttuso, La Vucciria
\end{artItem}

\begin{poem}
	\begin{stanza}
		che fregola mi scappa di mattina nel sesso\verseline
		di trovarmi voglia di meccaniche lusinghe\verseline
		d'inventarmi smania di dita\verseline
		imperiose serpi e maestre\verseline
		des études d'exécution transcendante\verseline
		che comandano miele alle membra
	\end{stanza}

	\begin{stanza}
		ma poi che fatica scrivere di cose\verseline
		indicibilissime\verseline
		di fame e di sete e di voglia\verseline
		di questo meccanico amore\verseline
		autoironico\verseline
		oh sì è anch'esso amore\verseline
		ne ha almeno la viltà\verseline
		e tanticchia di strazio
	\end{stanza}
\end{poem}

\clearpage

\poemtitle{iii}

\begin{poem}
	\begin{stanza}
		incerto il tuo gesto\verseline
		tra tenerezza e disprezzo\verseline
		ma forse\verseline
		non sapendo scegliere\verseline
		vuol tenere di una cosa\verseline
		e dell'altra
	\end{stanza}
\end{poem}

\clearpage

\poemtitle{iv}

\begin{artItem}
	Karel Teige, \begin{otherlanguage}{czech}%
		koláž číslo 355%
	\end{otherlanguage}
\end{artItem}

\begin{poem}
	\begin{stanza}
		come di capriolo già svegliato\verseline
		(che al mattino sa scrollarsi di dosso\verseline
		quanto di notturna rugiada gli si sia impigliata)\verseline
		sfregarsi in fretta il desiderio\verseline
		apprendendolo alla scorza di un pioppo
	\end{stanza}

	\begin{stanza}
		così non mi si appanna di struggimenti\verseline
		sùbiti lo sguardo per strada\verseline
		né mi sgretola un culo che passa la lealtà\verseline
		di buona bestia e chiusa
	\end{stanza}

	\begin{stanza}
		il capriolo che non vuol gravarsi di voglia\verseline
		quanto è lungo il giorno\verseline
		s'abbevera in fretta alla polla torbida\verseline
		pure se non lo trae la sete\verseline
		che non gliene salga più tardi l'assillo
	\end{stanza}

	\begin{stanza}
		e comunque anche nell’acqua sporca\verseline
		non resta sempre preso un po' di sole?
	\end{stanza}
\end{poem}

\clearpage

\poemtitle{v}

\begin{artItem}
	Oskar Kokoschka, \begin{otherlanguage}{german}%
		Nymphe%
	\end{otherlanguage}
\end{artItem}

\begin{poem}
	\begin{stanza}
		ma amo la venustà sottile che si cela\verseline
		nell'arco delle tue caviglie\verseline
		la tua pesantezza da odalisca\verseline
		che sale le scale
	\end{stanza}

	\begin{stanza}
		amo i seni che inutilmente copri\verseline
		nelle chiese (si volta anche il prete)\verseline
		ma che a me mostri indifferente\verseline
		al mio digiuno
	\end{stanza}

	\begin{stanza}
		dovrei allora amare ancora\verseline
		il tuo sesso? dimmi prima se puoi\verseline
		del mio sonno d'efebo verecondo\verseline
		cosa ne hai fatto?
	\end{stanza}
\end{poem}

\clearpage

\poemtitle{vi}

\begin{poem}
	\begin{stanza}
		spesso in questi giorni mi fa compagnia\verseline
		un oscillare da metronomo\verseline
		che millanta autenticità\verseline
		ma poi l'inflessibilità del moto\verseline
		conduce me al suo esito scontato\verseline
		e dimentica me lungo la via\verseline
		un grumo di umana angoscia ad attendermi\verseline
		però un'abluzione (quando accade)\verseline
		e un attenuante con me\verseline
		e si tira oltre
	\end{stanza}
\end{poem}

\clearpage

\poemtitle{vii}

\begin{poem}
	\begin{stanza}
		non portavo armatura\verseline
		non sai?
	\end{stanza}

	\begin{stanza}
		neppure un gregge di pecore\verseline
		avrei saputo guidare alla fonte\verseline
		o difendere dagli assalti dei lupi\verseline
		inetto di certo a montare\verseline
		il più bolso ronzino
	\end{stanza}

	\begin{stanza}
		perché m'investisti cavaliere\verseline
		contraria al destino?\verseline
		perché m'invitasti alla cura\verseline
		con gesto regale?
	\end{stanza}

	\begin{stanza}
		io non sapevo
	\end{stanza}
\end{poem}

\clearpage

\poemtitle{viii}

\begin{artItem}
	František Drtikol, fotografie
\end{artItem}

\begin{poem}
	\begin{stanza}
		poi mi va di trattenere dentro\verseline
		il desiderio come latte\verseline
		nel cavo delle mammelle\verseline
		o cadenza non risolta da tanto
	\end{stanza}

	\begin{stanza}
		e non mi va che mi coli via\verseline
		per le vie facili\verseline
		sicché comando alle mani\verseline
		di conoscere il tufo
	\end{stanza}

	\begin{stanza}
		e il mare riarso e la lava\verseline
		e il cinabro e insomma\verseline
		di sfiorare te prima\verseline
		di tornare a cercare me
	\end{stanza}
\end{poem}

\clearpage

\poemtitle{ix}

\begin{artItem}
	Giovanni Bellini, Sacra conversazione
\end{artItem}

\begin{poem}
	\begin{stanza}
		io non so più densa\verseline
		intensità di senso\verseline
		non dignità maggiore\verseline
		che in questo tuo fermarmi
	\end{stanza}

	\begin{stanza}
		nel gesto che dice la lotta\verseline
		ti ritrai ti concedi\verseline
		cedendo al desiderio\verseline
		a lui soltanto cedi terreno\verseline
		non a me d'un tratto spettatore
	\end{stanza}

	\begin{stanza}
		vuoi forse proteggermi?\verseline
		sottrarmi al pericolo\verseline
		di quale disfatta\verseline
		di giudiziosi propositi?\verseline
		temo sia tardi alquanto\verseline
		per questo
	\end{stanza}

	\begin{stanza}
		ma lascia che mi lasci guidare\verseline
		dalla maestà del tuo gesto\verseline
		e in esso appaghi e non appaghi\verseline
		il mio desiderio
	\end{stanza}

	\begin{stanza}
		lascia che vi scopra la gioia\verseline
		di stare nelle tue mani\verseline
		compiutamente
	\end{stanza}
\end{poem}

\clearpage

\poemtitle{x}

\begin{artItem}
	Renzo Vespignani, Gazometro
\end{artItem}

\begin{poem}
	\begin{otherlanguage}{french}
		\begin{stanza}
			j'ai appris que mes jours sont\verseline
			pareils à une éponge de Menger\verseline
			au-dessus d'un wagon plat
		\end{stanza}
	\end{otherlanguage}
\end{poem}

\clearpage

\poemtitle{xi}

\begin{artItem}
	Stefano Di Stasio, Dialettica della carne
\end{artItem}

\begin{poem}
	\begin{stanza}
		fermo, mi dici\verseline
		col gesto che intendo\verseline
		oltre lo schermo\verseline
		più che con parole
	\end{stanza}

	\begin{stanza}
		che io non richiami\verseline
		parlando di voglia\verseline
		tra noi il fantasma\verseline
		che già ci possiede
	\end{stanza}

	\begin{stanza}
		che non si consumi\verseline
		quel desiderio\verseline
		andando per le bocche\verseline
		degli altri
	\end{stanza}

	\begin{stanza}
		quel desiderio \verseline
		che non si nasconde\verseline
		che presiede al tuo: fermo,\verseline
		più che a invito palese
	\end{stanza}
\end{poem}

\clearpage

\poemtitle{xii}

\begin{poem}
	\begin{stanza}
		ci si disperde più volentieri\verseline
		in superbo autarchico trionfo\verseline
		che nel concedersi a invito di donna o di uomo
	\end{stanza}

	\begin{stanza}
		ma disperdersi è compromettersi\verseline
		è assentire all'imperfezione di un consumo frettoloso\verseline
		e io non ricordo fu mai senza lesioni il mio piacere
	\end{stanza}
\end{poem}

\clearpage

\poemtitle{xiii}

\begin{artItem}
	Egon Schiele, \begin{otherlanguage}{german}%
		Umarmung%
	\end{otherlanguage}
\end{artItem}

\begin{poem}
	\begin{stanza}
		il mio sesso è uno sterpo\verseline
		spezzato e secco\verseline
		e il tuo è un nido di rovi\verseline
		ma uniamoci lo stesso\verseline
		forse non ci saranno spasimi\verseline
		forse ci incastreremo
	\end{stanza}
\end{poem}

\clearpage

\poemtitle{xiv}

\begin{artItem}
	Alfred Kubin, \begin{otherlanguage}{german}%
		Das Erdrückende%
	\end{otherlanguage}
\end{artItem}

\begin{poem}
	\begin{stanza}
		ogni giorno conosco il desiderio\verseline
		che la passante getta per gioco\verseline
		conio genuino in un berretto da clochard
	\end{stanza}

	\begin{stanza}
		dal che risponde al turgore di lei\verseline
		pienezza inequivocabile nella mia carne\verseline
		so che è uguale alla mia la sua argilla
	\end{stanza}

	\begin{stanza}
		anzi\verseline
		so che lo fu\verseline
		ora \verseline
		è contorta da nasconderla, la mia,\verseline
		e da torcere lo sguardo che lei non ne rida
	\end{stanza}
\end{poem}

\clearpage

\poemtitle{xv}

\begin{artItem}
	Tristan Tzara, \begin{otherlanguage}{french}%
		Sans titre 1920%
	\end{otherlanguage}
\end{artItem}

\begin{poem}
	\begin{stanza}
		ti aggrumi\verseline
		nel mio dormiveglia affannato\verseline
		ti adorni di sostanza carnale\verseline
		di presenza che non so ignorare
	\end{stanza}

	\begin{stanza}
		sei scomoda\verseline
		le tue morbidezze celano spigoli\verseline
		ti sveli inghiottitrice di giorni\verseline
		e mi parli incongrua e stridente\verseline
		nella tana infeconda del letto
	\end{stanza}

	\begin{stanza}
		e mi abbaglia\verseline
		più chiara del sole la persuasione atroce\verseline
		che non ci saranno tempi di luce\verseline
		che è già parato il veleno
	\end{stanza}

	\begin{stanza}
		ma mi tradisce il risveglio\verseline
		e ora non so ritrovare il perché\verseline
		e mi lascio aggrovigliare\verseline
		nel tuo mistero
	\end{stanza}

	\begin{stanza}
		ancora
	\end{stanza}
\end{poem}

\clearpage

\poemtitle{xvi}

\begin{artItem}
	Egon Schiele, \begin{otherlanguage}{german}%
		Selbstbildnis mit rückwärtsgehaltenen Armen%
	\end{otherlanguage}
\end{artItem}

\begin{poem}
	\begin{otherlanguage}{french}
		\begin{stanza}
			ceci n'est que faire\verseline
			de l'archéologie\verseline
			je dit se branler\verseline
			par cœur
		\end{stanza}

		\begin{stanza}
			ce désir en revanche\verseline
			d'aujourd'hui\verseline
			je ne veux pas qu'il gagne\verseline
			la tchernaïa zemlia charnelle\verseline
			avant\verseline
			(pourvu qu'on franchît cela)\verseline
			qu'on nique
		\end{stanza}
	\end{otherlanguage}
\end{poem}

\clearpage

\poemtitle{xvii}

\begin{artItem}
	Enrico Castellani, Superficie rossa
\end{artItem}

\begin{poem}
	\begin{stanza}
		fioritura di sangue\verseline
		in cristalli sapidi\verseline
		tesoro a milioni\verseline
		di tornesi pompeiani
	\end{stanza}

	\begin{stanza}
		colmarne la conca delle mani\verseline
		impiastricciarsi di sangue\verseline
		o mangiarseli uno a uno\verseline
		calibrando il capriccio?
	\end{stanza}

	\begin{stanza}
		è gioco da equilibrista\verseline
		non lasciarne cadere\verseline
		uno neppure dei grani\verseline
		di prosapia fenicia
	\end{stanza}

	\begin{stanza}
		a pena di sbucciare\verseline
		dai petali ardenti\verseline
		onde lo si sigillava\verseline
		un pezzetto di mondo
	\end{stanza}
\end{poem}

\clearpage

\poemtitle{xviii}

\begin{poem}
	\begin{stanza}
                guizzi improvvisi\verseline
                eccitazione e mollezze\verseline
                flaccida cruda carne si erge\verseline
                ricade
	\end{stanza}

	\begin{stanza}
                automatici gesti\verseline
                a placare così un vuoto che divora\verseline
                ma si denuncia da sé l’aporia
	\end{stanza}

	\begin{stanza}
                eppure non si saprebbe fermarsi prima d’aver concluso il lavoro
	\end{stanza}

	\begin{stanza}
                ed ha la meglio alla fine\verseline
                un magro corpo d’adolescente\verseline
                la sua lascivia tenera\verseline
                e la tenerezza d’un cedimento
	\end{stanza}

	\begin{stanza}
                effondersi quieti a onta delle carezze grevi
	\end{stanza}
\end{poem}

\clearpage

\poemtitle{xix}

\begin{poem}
	\begin{stanza}
                perché cerchiamo una presenza facile\verseline
                rassicurante\verseline
                un piatto lasciato in caldo\verseline
                un rifugio sicuro e parole amiche?
	\end{stanza}

	\begin{stanza}
                suvvia\verseline
                è ora di andare
	\end{stanza}

	\begin{stanza}
                ma non tale è la vita non può\verseline
                troppo ostentatamente prodiga\verseline
                offrirci un asilo sicuro e un letto sincero
	\end{stanza}

	\begin{stanza}
                troppo abbiamo indugiato\verseline
                bisogna andare\verseline
                al nostro amore consacreremo un pezzetto di ricordo\verseline
                al riparo dal sole
	\end{stanza}

	\begin{stanza}
                quanto a noi\verseline
                altre lusinghe fermeranno lo sguardo
	\end{stanza}
\end{poem}

\clearpage

\poemtitle{xx}

\begin{poem}
	\begin{stanza}
                certi rifugi scontrosi nella carne\verseline
                ti allettano con la promessa di un frammento\verseline
                di vita discreto e a buon mercato
	\end{stanza}

	\begin{stanza}
                ti lasci sedurre
	\end{stanza}

	\begin{stanza}
                ti lasci prendere a poco a poco\verseline
                nella ciclicità dorata e inesorabile\verseline
                che li sottende
	\end{stanza}

	\begin{stanza}
                ti lasci distogliere\verseline
                (con sollievo)\verseline
                dalla vita\verseline
                quella vera\verseline
                salata\verseline
                spigolosa
	\end{stanza}
\end{poem}

\clearpage

\poemtitle{xxi}

\begin{artItem}
	Hieronymus Bosch, \begin{otherlanguage}{dutch}%
		Tuin der lusten%
	\end{otherlanguage}
\end{artItem}

\begin{poem}
	\begin{stanza}
		perché ruba il fiato per un attimo\verseline
		il risucchio del vuoto\verseline
		dopo uno spasimo freddo\verseline
		dopo un abbaglio di pixel?\verseline
		ma giusto un attimo che tutto è ok\verseline
		si torna all'ordinata gestione\verseline
		scarichi ripuliti ipocritamente\verseline
		sorridenti alla vita
	\end{stanza}

	\begin{stanza}
		eppure anche di questo siamo fatti\verseline
		di voler strappare il senso\verseline
		all'incontro con l'altro\verseline
		senza incontrare nessuno\verseline
		di voler chiudere nel pugno il mistero\verseline
		di voler fare autarchica norma\verseline
		il bastare a noi stessi
	\end{stanza}

	\begin{stanza}
		eppure anche di questo siamo fatti\verseline
		di dimestichezza con le viltà della carne\verseline
		quando travolta irritata\verseline
		si ritira in buon ordine\verseline
		avendo concesso di sé al mondo\verseline
		non più che un infimo parsimonioso\verseline
		sporcarsi
	\end{stanza}
\end{poem}

\clearpage

\poemtitle{xxii}

\begin{artItem}
	\begin{otherlanguage}{german}%
		Niederrheinischer Meister, Liebeszauber%
	\end{otherlanguage}
\end{artItem}

\begin{poem}
	\begin{stanza}
		quei calici gemelli\verseline
		di vino misturato\verseline
		un po' stanchi dal peso\verseline
		io so a che somigliarli\verseline
		ma preferisco tacerlo\verseline
		non sta bene il parlare indecente
	\end{stanza}
\end{poem}

\clearpage

\poemtitle{xxiii}

\begin{artItem}
	Pollock, \begin{otherlanguage}{english}%
		Number 32%
	\end{otherlanguage}
\end{artItem}

\begin{poem}
	\begin{stanza}
		che malinconia dopo\verseline
		quando nella cenere si spengono i bengala\verseline
		quando rancidiscono le vivande appetitose
	\end{stanza}

	\begin{stanza}
		per dieci secondi il mondo è sporco
	\end{stanza}

	\begin{stanza}
		ma poi basta un recto in 16º per nettare\verseline
		la superficie immacolata del giorno
	\end{stanza}
\end{poem}

\clearpage

\poemtitle{xxiv}

\begin{poem}
	\begin{stanza}
		venisti qui per un gioco leggero\verseline
		per un giorno o due di carezze\verseline
		ma poi sei rimasta\verseline
		accoccolata tra le lenzuola\verseline
		come bestiola senza difese
	\end{stanza}

	\begin{stanza}
		non erano questi i taciti patti
	\end{stanza}
\end{poem}

\clearpage

\poemtitle{xxv}

\begin{artItem}
	George Frederic Watts, \begin{otherlanguage}{english}%
		The Minotaur%
	\end{otherlanguage}
\end{artItem}

\begin{poem}
	\begin{stanza}
		l'inferno lo si riconosce soltanto\verseline
		al percorrerlo aggrappati\verseline
		con le mani al sesso\verseline
		come a un plettro di lira\verseline
		o a un fuso intrecciato di spago
	\end{stanza}
\end{poem}

\clearpage

\poemtitle{xxvi}

\begin{artItem}
	Balthus, \begin{otherlanguage}{french}%
		La Rue%
	\end{otherlanguage}
\end{artItem}

\begin{poem}
	\begin{stanza}
		quando ci accorgiamo che ci serve\verseline
		un testimone per tutti i giorni\verseline
		lo troviamo in terre familiari\verseline
		compimento di un solco tracciato
	\end{stanza}

	\begin{stanza}
		poi però andiamo fuori a cercarci\verseline
		voglia di ribellione e paura\verseline
		e desiderio\verseline
		da vivere o da sentircene un attimo\verseline
		appena sfiorati
	\end{stanza}
\end{poem}

\clearpage

\poemtitle{xxvii}

\begin{artItem}
	Jan Weenix, \begin{otherlanguage}{dutch}%
		Dode zwaan en pauw voor een lusthof%
	\end{otherlanguage}
\end{artItem}

\begin{poem}
	\begin{stanza}
		favo di miele\verseline
		gonfio appiccicoso\verseline
		dai riflessi del vetro\verseline
		come esce dal forno\verseline
		e poi è preso\verseline
		caldo sul ferro
	\end{stanza}

	\begin{stanza}
		il tuo volto ora è maschera\verseline
		antica di un dio\verseline
		di un atleta\verseline
		impetrato da morte\verseline
		lo sguardo appeso a un punto\verseline
		dietro i miei occhi
	\end{stanza}

	\begin{stanza}
		mi faccio serio e ti rendo\verseline
		il tributo che non si nega\verseline
		a chi cade sconfitto nell'arena\verseline
		prima di piegarti alla\verseline
		inevitabile resa
	\end{stanza}
\end{poem}

\clearpage

\poemtitle{xxviii}

\begin{artItem}
	Salvador Dalì, \begin{otherlanguage}{spanish}%
		La verdadera pintura de la «Isla de los Muertos» de Arnold Böcklin en la hora del Ángelus%
	\end{otherlanguage}
\end{artItem}

\begin{poem}
	\begin{stanza}
		scivolo in cedimenti troppo noti\verseline
		perché mi facciano più tremare\verseline
		e preferisco acconsentire\verseline
		al veleno sottile delle giornate
	\end{stanza}

	\begin{stanza}
		meglio imparare dolcemente\verseline
		meglio lasciarsi cadere piano\verseline
		issare il vessillo della resa\verseline
		scendere a patti con tagliole incustodite
	\end{stanza}

	\begin{stanza}
		ma non so che pensare\verseline
		poiché so che così mi lascio scorrere tra le dita\verseline
		l'urgenza della mia scrittura\verseline
		e l'impossibilità di tacere
	\end{stanza}
\end{poem}

\clearpage

\poemtitle{xxix}

\begin{poem}
	\begin{stanza}
		che questo incontro\verseline
		non sperato non atteso\verseline
		ma soltanto\verseline
		studiatamente preparato\verseline
		trascorra in fretta\verseline
		come va tra la folla\verseline
		il commesso affannato\verseline
		alla stazione
	\end{stanza}
\end{poem}

\clearpage

\poemtitle{xxx}

\begin{poem}
	\begin{stanza}
		su raccogliamo le some\verseline
		di voglia e di colpa\verseline
		e concediamo asilo\verseline
		alla materia più fetida\verseline
		alla materia oscura del vivere\verseline
		quella che ai fortunati è permesso\verseline
		e agli ipocriti sdegnosi\verseline
		di gettare via tra le cose\verseline
		di cui non è bello parlare\verseline
		e che ci sia lieve il fardello\verseline
		per quanto possibile
	\end{stanza}
\end{poem}

\clearpage

\poemtitle{xxxi}

\begin{poem}
	\begin{stanza}
		continua a bruciare\verseline
		la fredda scambievole violenza che ci tiene insieme\verseline
		che a me strappa le ore minuto a minuto\verseline
		a me consapevole\verseline
		a me di marmo sanguinante\verseline
		come il giocatore incatenato al tavolo che lo spoglia
	\end{stanza}

	\begin{stanza}
		ma alla nuova se ci sarà io chiedo\verseline
		gli occhi senza sguardo\verseline
		i passi serrati a un binario\verseline
		che corre presso alla frontiera delle mie cose\verseline
		e una cura perentoria\verseline
		e giorni di angoscia q.b.
	\end{stanza}
\end{poem}

\clearpage

\poemtitle{xxxii}

\begin{artItem}
	Egon Schiele, \begin{otherlanguage}{german}%
		Nach vorn gebeugter weiblicher Akt%
	\end{otherlanguage}
\end{artItem}

\begin{poem}
	\begin{stanza}
		non ignaro del mio dovere\verseline
		mi costringo a tendere i sensi\verseline
		che ne sia appagato quello sguardo\verseline
		oltre la schiena salda al mio fianco
	\end{stanza}

	\begin{stanza}
		ma non so incoccare la freccia all'arco\verseline
		né so adattare il plettro alla lira\verseline
		perché il mio desiderio è andato\verseline
		dove non posso raggiungerlo
	\end{stanza}
\end{poem}

\clearpage

\poemtitle{xxxiii}

\begin{poem}
	\begin{stanza}
		cosa fummo\verseline
		se non due cammini\verseline
		intrecciati in terra straniera\verseline
		se non due ruscelli giovani\verseline
		geologie attraversate di nascosto alla luce\verseline
		poi gettati sopra la terra, insieme?
	\end{stanza}

	\begin{stanza}
		denunciati senza preavviso al sole\verseline
		stupore liquido nel colmo del giorno\verseline
		di specchiarci l'un l'altro\verseline
		nel corso gemello\verseline
		ribellione a scoprire\verseline
		ancora due solchi disgiunti
	\end{stanza}

	\begin{stanza}
		cosa fummo\verseline
		se non due uccelli giovani\verseline
		sporchi ancora di madre?\verseline
		ali da provare e una spezzata di volo\verseline
		a beccarci di gusto crudele\verseline
		a tagliarci l'un l'altro ogni guizzo di fuga
	\end{stanza}

	\begin{stanza}
		cosa fummo\verseline
		se non due ragazzi giovani\verseline
		il tempo scoperto d'un tratto\verseline
		mesi da correre d'un fiato\verseline
		il mondo rifatto nuovo e nostro?
	\end{stanza}
	
	\begin{stanza}
		e non volerne più sapere\verseline
		o non ancora\verseline
		di anni diversi e più vili\verseline
		da soli
	\end{stanza}
\end{poem}

\clearpage

\poemtitle{xxxiv}

\begin{artItem}
	Carlo Carrà, L’amante dell’ingegnere
\end{artItem}

\begin{poem}
	\begin{stanza}
		presenza che non so\verseline
		se tu sia più pervasiva o più accogliente\verseline
		corpo chimerico opalescente traslucido\verseline
		da alterarne il tempo e lo spazio\verseline
		come massiva stella\verseline
		lasciala intatta la geometria del mio mondo\verseline
		ché ho bisogno di una percezione spigolosa\verseline
		non di te molle e accogliente
	\end{stanza}
\end{poem}

\clearpage

\poemtitle{xxxv}

\begin{poem}
	\begin{stanza}
		dove la pienezza mischiata\verseline
		di sospensione angosciosa\verseline
		dove la pienezza antica\verseline
		odorosa di seme non estraneo?
	\end{stanza}

	\begin{stanza}
		ora l’angoscia insistente\verseline
		bordone alla vita di un uomo\verseline
		grumo che volta a volta\verseline
		si contrae a ogni effusione
	\end{stanza}

	\begin{stanza}
		nulla cambia a questo seme\verseline
		che a sollecitarlo\verseline
		o e ad accoglierlo\verseline
		sia carne altrui
	\end{stanza}
\end{poem}

\clearpage

\poemtitle{xxxvi}

\begin{poem}
	\begin{stanza}
		che siamo ora, mia diletta?\verseline
		anche se fuggono i giorni\verseline
		e per scherno ha violato il tempo\verseline
		i bei corpi (orgoglio di giovinezza)\verseline
		non altro saprei trovarmi intorno\verseline
		che valga i tuoi seni appassiti
	\end{stanza}
\end{poem}

\clearpage

\poemtitle{xxxvii}

\begin{artItem}
	Maurits Cornelis Escher, \begin{otherlanguage}{dutch}%
		Spiralen%
	\end{otherlanguage}
\end{artItem}

\begin{poem}
	\begin{otherlanguage}{french}
		\begin{stanza}
			dès le matin décharger vite\verseline
			avant de s'habiller pour la journée
		\end{stanza}
	\end{otherlanguage}
\end{poem}
