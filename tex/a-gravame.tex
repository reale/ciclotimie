\chapter*{A gravame}

\poemtitle{i}

\begin{verse}
    mi interrogano\\
    e io so solo intridermi\\
    di umidità fetente\\
    ogni fibra erta a nascondere\\
    il putrido di dentro\\
    degno di sprezzo
\end{verse}

\begin{verse}
    che non s'accorgano della paura\\
    che porto dentro\\
    solo questo mi empie la testa\\
    che non s'accorgano della paura\\
    che non so strapparmi di dentro
\end{verse}

\clearpage

\poemtitle{ii}

\begin{verse}
    impalpabile come velo di sposa\\
    ovunque io mi volga\\
    qualcosa che non so definire\\
    intercetta la mia mano e il mio sguardo
\end{verse}

\begin{verse}
    e l'abbraccio virile d'un amico\\
    e la carezza dell'amante\\
    egualmente si spengono, quando mi raggiungono\\
    attraverso l'ignota membrana
\end{verse}

\begin{verse}
    come le parole, che affidate alla voce\\
    o alla penna inesperta\\
    non sanno gettare al vento\\
    la pesante zavorra, odiosa
\end{verse}

\begin{verse}
    ma destiamoci, e squarciamo\\
    il velo tenace, infecondo
\end{verse}

\begin{verse}
    che non resti inespresso il messaggio\\
    che non giunga sigillato ancora\\
    alle porte dell'Ade
\end{verse}

\begin{verse}
    che non trascorrano i giorni\\
    consumati interi nella finzione\\
    d'una ribellione tiepida
\end{verse}

\clearpage

\poemtitle{iii}

\begin{verse}
    che pensavi salendo all'orto?
\end{verse}

\begin{verse}
    ti premevano il cuore le sorti dell'uomo?
\end{verse}

\begin{verse}
    non è che invece\\
    ti incantavano le luci della città\\
    il formicolio scuro di vita carnale?
\end{verse}

\begin{verse}
    dimmi, che pensavi?
\end{verse}

\begin{verse}
    t'attendeva strazio... te ne curavi?\\
    prelibavi forse lo schiaffo del guitto\\
    la carne stracciata\\
    pregiato e raro esperire?\\
    o il repertorio intero dei lazzi\\
    dal terrore starnazzante\\
    all'indifferenza esibita con arte?
\end{verse}

\begin{verse}
    che pensavo io salendo all'orto?\\
    barattare il resto dei giorni\\
    per un pugno di consapevolezza\\
    dura come concrezioni di perla in grembo di donna\\
    è da crepuscolari dèi\\
    non da mortali
\end{verse}

\clearpage

\poemtitle{iv}

\begin{verse}
    dormivano, diranno: dormivano\\
    ma io non dormo\\
    né dormono i compagni
\end{verse}

\begin{verse}
    soltanto che dolcezza\\
    tenere chiusi gli occhi\\
    le membra colte nella tunica\\
    strette alla terra
\end{verse}

\begin{verse}
    dentro gli occhi stanno ancora\\
    le piazze piene della capitale\\
    i fianchi pieni delle donne
\end{verse}

\begin{verse}
    lui non dorme\\
    tiene gli occhi aperti\\
    e le spalle piegate
\end{verse}

\begin{verse}
    è lontana da qui la mia casa
\end{verse}

\begin{verse}
    che dolcezza sapere che non si può cambiare il mondo\\
    che la vita ti tradisce\\
    è dolce anche la stanchezza della disfatta\\
    e noi sapevamo di essere esperti della vita\\
    più di lui\\
    che non guardava i fianchi delle donne
\end{verse}

\begin{verse}
    di lui\\
    che non dorme\\
    e tiene gli occhi aperti\\
    si sa\\
    e forse viene a cercarci
\end{verse}

\begin{verse}
    ma noi\\
    anche se tenevamo soltanto gli occhi chiusi\\
    un poco\\
    e il corpo stretto alla terra\\
    diranno che dormivamo
\end{verse}

\clearpage

\poemtitle{v}

\begin{verse}
    cosa vedesti di là dalle colonne?
\end{verse}

\begin{verse}
    una fuga infinita di mondi\\
    appena oltre il ciglio dell'abisso\\
    che forse ti sembrò bastare\\
    sporgere la mano per toccarli?
\end{verse}

\begin{verse}
    o forse soltanto ti balenò la paura\\
    indomata\\
    che mai abbandona le radici di ogni vita mortale?\\
    fu essa a sedurti?
\end{verse}

\begin{verse}
    andasti
\end{verse}

\clearpage

\poemtitle{vi}

\begin{verse}
    lascia stare compagno\\
    il compagno caduto\\
    lascialo appeso\\
    ai corvi
\end{verse}

\begin{verse}
    portare a sua madre\\
    un corpo sconfitto\\
    con le parole mozzate\\
    o alla terra?
\end{verse}

\begin{verse}
    ora le tue braccia sono quelle\\
    di una vecchia prosciugata vergine zia\\
    che tiene il bimbetto di sua sorella\\
    braccia dure di legno verde
\end{verse}

\begin{verse}
    tu questa carne\\
    senza più parola\\
    la nascondi\\
    nella terra
\end{verse}

\begin{verse}
    tu non sai quanti viaggi\\
    ancora\\
    per tutti quegli altri\\
    compagni
\end{verse}

\clearpage

\poemtitle{vii}

\begin{verse}
    non so che mi sgorga\\
    se sangue o acqua dal costato\\
    regolarmente trafitto\\
    o seme giovane dal sesso\\
    né so che mi stira\\
    le membra (antiche?)\\
    no,\\
    ma tenere, e pegno di memoria
\end{verse}

\begin{verse}
    se mi insanguina la gola\\
    grido di carne offesa,\\
    o, a non saper dirlo,\\
    solo mi sforza la voce\\
    carne accesa di primavera;\\
    non so anche questo, e non so\\
    se siano in me lo stesso
\end{verse}

\clearpage

\poemtitle{viii}

\begin{verse}
    ma farmi sapere al grido\\
    che non risponde al mio ancora\\
    di voci attese da tanto\\
    che gli hanno detto di lasciare\\
    a casa il loro odore di carne\\
    e non amano perciò che lacrime di cera\\
    e spregiano i miei succhi e gridano\\
    nello stormire insopportabile\\
    di cosa? forse ulivi nella notte
\end{verse}

\clearpage

\poemtitle{ix}

\begin{verse}
    comunque mi perdo nel canto\\
    di voci di carne comunque,\\
    di passione per dovere, ed è uguale\\
    offro, ne volete?, sangue pregiato\\
    da ornare le vostre labbra di carne\\
    e seme da stillarne sui vostri seni\\
    e ritrosia di puledro nascostamente storpio\\
    a divertire la vostra lucentezza\\
    e gemiti, pochi, che tolgo al padre
\end{verse}

\begin{verse}
    per me mi lascio un tremito solo mio
\end{verse}

\clearpage

\poemtitle{x}

\begin{verse}
    alle luci dell'alba, alla strada già chiara\\
    alle torce (inutili) tra le mani\\
    al voltare della cantonata\\
    con la via che slarga verso il mare\\
    al troppo\\
    lezzo di primavera che comincia\\
    (finalmente)\\
    e ai tuoi seni barbari, per ultimi
\end{verse}

\clearpage

\poemtitle{xi}

\begin{verse}
    il confine tra veglia chiacchierina\\
    e sonno inconsapevole\\
    l'ho superato stanotte mille volte\\
    come un fiume basso da passare a guado\\
    in un senso prima e poi nell'altro
\end{verse}

\begin{verse}
    se ai miei occhi ogni frontiera\\
    non è mai netta indiscutibile\\
    ma frastagliata e gonfia e terra ampia\\
    è alle notti agitate e alla contesa aperta\\
    di sonno e veglia che ne son debitore
\end{verse}

\clearpage

\poemtitle{xii}

\begin{verse}
    ancora mi seduce la sostanza oscura\\
    e calda delle notti sicché corro\\
    a rifugiarmi tra le coltri\\
    preso da voglia di tana
\end{verse}

\begin{verse}
    e che siano serrate a doppia mandata\\
    le porte\\
    che non vi s'insinui\\
    agonia viscida d'inverno
\end{verse}

\begin{verse}
    e non è più il tempo che mi lascio\\
    strappare al sonno\\
    da voci trascinate nel buio\\
    ma che non lo vorrei ancora una volta\\
    non chiedetemi giuramento
\end{verse}

\clearpage

\poemtitle{xiii}

\begin{verse}
    cari muri d'edera e vecchi lampioni\\
    quante volte mi foste quinta\\
    a un vagare interminato\\
    compagni nell'imbarazzo del mattino\\
    scorta di strada senza fine
\end{verse}

\begin{verse}
    ma non lasciate amici\\
    che io sia solo stanotte\\
    negli angoli gremiti di rottami\\
    si appiatta il corteo dei giorni\\
    l'affanno mi toglie le forze\\
    il non saper dire\\
    ancora\\
    sono nascosti\\
    laggiù
\end{verse}

\clearpage

\poemtitle{xiv}

\begin{verse}
    solo gesto di ribellione\\
    offrirsi alle raffiche nudi\\
    respirare aria secca tagliente\\
    lasciarsi scovare qualcosa\\
    da qualche parte
\end{verse}

\begin{verse}
    ma poi ieri ho sentito il calore degli amici\\
    per la prima volta\\
    non lasciarsi squarciare il petto\\
    dalla durezza degli sguardi
\end{verse}

\clearpage

\poemtitle{xv}

\begin{verse}
    non è ancora allentata\\
    la morsa chiusa dell'inverno\\
    ma lo sarà presto
\end{verse}

\begin{verse}
    come un esercito schierato\\
    come le schiere rosse dell'armata fuori dai reticolati\\
    così è la primavera tutt'intorno alle mura\\
    e manda lontano i canti di guerra e le grida
\end{verse}

\begin{verse}
    io non sbircio dalle finestre\\
    ma spingo un braccio oltre gli interstizi della tana
\end{verse}

\clearpage

\poemtitle{xvi}

\begin{verse}
    nient'altro che\\
    un contorcersi di nero nel nero\\
    però un contorcersi regale\\
    e artificiali lucciole che non si contavano\\
    e la solita notte incorente\\
    la solita notte in problematico equilibrio\\
    la solita notte che non dà pace
\end{verse}

\clearpage

\poemtitle{xvii}

\begin{verse}
    si strappa il velo del cielo\\
    si strappa ad oriente la tregua della notte\\
    troppo tardi per negarsi alla crudeltà primaverile\\
    troppo tardi per aggrapparsi all'inverno che dilegua\\
    e il giorno dilaga tra le pietre\\
    e per le vie di giovinezza antica\\
    le membra stracche dal carico inutile e lungo\\
    si riposano\\
    e non si risparmiano gli schiaffi sulle spalle dei compagni\\
    che ormai la fatica è compiuta\\
    e poi a stare tutti insieme si sente meno\\
    questo giorno che comincia stupito e senza ombre\\
    fiaccole opache e inutili\\
    estinte a una a una\\
    con gesto di carnefice
\end{verse}

\clearpage

\poemtitle{xviii}

\begin{verse}
    enfia inquieta\\
    litania di piaghe\\
    lavate di mare\\
    fragranti di sangue\\
    asperse di sale
\end{verse}

\begin{verse}
    di fere quel sanguine\\
    di mostri pelagici\\
    mogliere lubriche\\
    o di carni stillanti\\
    del corpo del nume
\end{verse}

\begin{verse}
    ecce homo\\
    ma della terra tradita\\
    chi sa le ulcere?
\end{verse}

\begin{verse}
    e il sole appannato\\
    e l'aria che assorbe il morbo\\
    chi li dice?
\end{verse}

\clearpage

\poemtitle{xix}

\begin{verse}
    c'è dentro più di durezza\\
    negli artigli d'astore dell'inverno\\
    aggrappati alle carni\\
    o nel dilatarsi del buio\\
    di questa notte?
\end{verse}

\begin{verse}
    s'immagina lontana l'alba\\
    ma non è
\end{verse}

\begin{verse}
    un gallo riempie la cavità della notte\\
    di un gracchiare fuori tempo\\
    forse per la troppo incongrua\\
    dilatazione
\end{verse}

\begin{verse}
    è fresco
\end{verse}

\begin{verse}
    il lenzuolo un cencio lasciato in un angolo\\
    il gallo neppure fa più paura\\
    sta lì muto o stride ogni tanto
\end{verse}

\clearpage

\poemtitle{xx}

\begin{verse}
    non è ancora\\
    o non è più il tempo\\
    di prestare orecchio\\
    a un remoto dolore carnale\\
    eppure\\
    troppo di me c'è rimasto incastrato\\
    onde il continuo riandare
\end{verse}

\clearpage

\poemtitle{xxi}

\begin{verse}
    sgorgare di voci\\
    rorida messe\\
    di spighe feconde
\end{verse}

\begin{verse}
    sensi tesi\\
    incerta sapienza\\
    disperazione carnale
\end{verse}

\begin{verse}
    memorie di una\\
    perduta pubertà\\
    passione voluta gridare
\end{verse}

\begin{verse}
    ma soffocata nella carne\\
    un passo prima di primavera
\end{verse}

\clearpage

\poemtitle{xxii}

\begin{verse}
    vessillo esposto e spregiato\\
    panno mondato di macchie\\
    lenzuolo tradito di nozze
\end{verse}

\begin{verse}
    schiocchi appiccicosi\\
    come sofferenza dolce\\
    vessillo teso nell'aria\\
    tesa essa pure di freddo\\
    che va via
\end{verse}

\begin{verse}
    sera lontana di primavera\\
    lontana sera di strazio\\
    di vento e di sensi\\
    di dolorosa nascita\\
    di remota venuta al mondo\\
    scaglia di notte più densa\\
    straccio tessuto troppo in fretta\\
    telo riempito di paure\\
    cortina spregiata
\end{verse}

\begin{verse}
    messe di vento\\
    forma di geometria non dedotta\\
    incalzare non placato\\
    non ancora
\end{verse}

\clearpage

\poemtitle{xxiii}

\begin{verse}
    giorno pieno
\end{verse}

\begin{verse}
    per la finestra spalancata\\
    si versa dentro alla stanza\\
    il cielo di nuvolaglia
\end{verse}

\begin{verse}
    un brandello di sonno ancora\\
    strappato agli uffici mattinali
\end{verse}

\begin{verse}
    è il sogno del padre\\
    finalmente remoto\\
    (scoppio d'odio o di passione tardiva?)\\
    l'estrema difesa
\end{verse}

\begin{verse}
    risvegliarsi poi alla ferraglia\\
    del tram\\
    che riempie la strada
\end{verse}
