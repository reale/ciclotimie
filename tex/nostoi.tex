\begin{volumetitlepage}
	\volumetitle{Nostoi}
	\volumeheader{Nostoi}
\end{volumetitlepage}

\poemtitle{\foreignlanguage{french}{mœurs romaines}}

\begin{artItem}
	Kurt Schwitters, \begin{otherlanguage}{german}%
		Zollamtlich geöffnet%
	\end{otherlanguage}
\end{artItem}

\begin{poem}
	\begin{otherlanguage}{french}
		\begin{stanza}
			entre chien et loup\verseline
			les tentacules de béton\verseline
			coulissant au milieu\verseline
			des blés impudiques\verseline
			jamais ne s'empêchent\verseline
			de dégager leurs rayures\verseline
			en bitume allongées
		\end{stanza}
	\end{otherlanguage}
\end{poem}

\clearpage

\poemtitle{notturno}

\begin{poem}
	\begin{stanza}
		quando ti sogno\verseline
		mia capitale\verseline
		sempre ti so assediata di sonno\verseline
		strapiena di strade\verseline
		disfatta
	\end{stanza}

	\begin{stanza}
		(mia te lo dico\verseline
		ad colorandam possessionem\verseline
		per lo struggimento delle tue piazze\verseline
		precipue de nocte\verseline
		quod oculis nocet)
	\end{stanza}

	\begin{stanza}
		formicolante d’ombre\verseline
		amiche a chi sa bucarle\verseline
		dimora a randagi come allumano i lampioni\verseline
		colante dal tuo centro\verseline
		pigiatura matura d’uve nell’imbuto
	\end{stanza}
\end{poem}

\clearpage

\poemtitle{vicoli}

\begin{artItem}
	Pina Calì, Galline
\end{artItem}

\begin{poem}
	\begin{stanza}
		i vicoli all'alba\verseline
		sono campi seminati\verseline
		di rebus accartocciati\verseline
		di scaglie di luce\verseline
		di leccornie esanimi\verseline
		arengo di gabbiani
	\end{stanza}
\end{poem}

\clearpage

\poemtitle{laurentino 38}

\begin{poem}
	\begin{stanza}
		sei confine, grumo rappreso di cemento\verseline
		sull'orlo dell'agro, intrico di gusci
	\end{stanza}

	\begin{stanza}
		sei gioco di forme, finezza sospesa\verseline
		di macchina
	\end{stanza}

	\begin{stanza}
		sei sogno di logos orgoglioso
	\end{stanza}

	\begin{stanza}
		sei ingenuità, sei indugio, sei concrezione
		lenta a crescere
	\end{stanza}

	\begin{stanza}
		sei le pastoie del burocrate, e le ragioni\verseline
		troppo brevi della politica
	\end{stanza}

	\begin{stanza}
		sei colpa
	\end{stanza}

	\begin{stanza}
		sei vita non voluta, vita di gramigna\verseline
		che ti fa suo
	\end{stanza}

	\begin{stanza}
		sei enclave, ricetto di guai, terra mala
	\end{stanza}

	\begin{stanza}
		sei coagulo di desiderio
	\end{stanza}

	\begin{stanza}
		sei tu (dicono) a guardare di laggiù\verseline
		l’E42\verseline
		con sguardo da escluso\verseline
		bramoso\verseline
		di schiavo
	\end{stanza}
\end{poem}

\clearpage

\poemtitle{treni a Portonaccio}

\begin{poem}
	\begin{stanza}
		chi non sa gli arrivi e le partenze\verseline
		e le manovre e le macchine fumanti\verseline
		e i lamenti da gabbiano dei freni\verseline
		e l'ordine saldo e onesto del capo\verseline
		chi non sa tutto questo\verseline
		non ti conosce, Roma
	\end{stanza}

	\begin{stanza}
		e lasciamogli pure quelle pietre\verseline
		consunte dal gregario\verseline
		a chi di te non guarda\verseline
		che i fasti antichi
	\end{stanza}

	\begin{stanza}
		noi amiamo il vapore nelle notti d'inverno\verseline
		e la fatica uguale di macchine e uomo\verseline
		la dignità tua nuova\verseline
		la tua vita di metallica fenice
	\end{stanza}
\end{poem}

\clearpage

\poemtitle{autunnale per san lorenzo}

\begin{artItem}
	Lyonel Feininger, \begin{otherlanguage}{german}%
		Barfuesserkirche I%
	\end{otherlanguage}
\end{artItem}

\begin{poem}
	\begin{stanza}
		chi sa la tua controra di quartiere\verseline
		romana tanto poco e il tuo\verseline
		antimediterraneo pan\verseline
		che gioca a rimpiattino\verseline
		per strade ferrate e fabriche\verseline
		vòlte a settentrione?
	\end{stanza}

	\begin{stanza}
		chi sa la nostalgia d'un pomeriggio\verseline
		lieve di domenica d'autunno\verseline
		che sono spente le tède\verseline
		—ma appena appena—\verseline
		delle tue ebdomadarie tesmoforie\verseline
		e già si danno per sedotte\verseline
		a un raziocinio ossidato e caldo\verseline
		le tue strade?
	\end{stanza}

	\begin{stanza}
		chi sa di quanto tu infittisci\verseline
		la trama della mia appartenenza\verseline
		sarcitura di opera e di umori?
	\end{stanza}
\end{poem}

\clearpage

\poemtitle{casilino 23}

\begin{artItem}
	Mario Sironi, Paesaggio urbano con manichino
\end{artItem}

\begin{poem}
	\begin{stanza}
		la nebbia—quel che resta\verseline
		alla terra di agro scomparso—\verseline
		si lascia cucire da fili\verseline
		di lampioni senza ombrarsi
	\end{stanza}

	\begin{stanza}
		io mi ci rannicchio appena\verseline
		a un passo anche stanotte\verseline
		dalle falangi enormi\verseline
		sul ciglio della messe\verseline
		di gioie domestiche\verseline
		a decine di migliaia
	\end{stanza}

	\begin{stanza}
		e voglio che il mio luogo\verseline
		—questo tra i miei luoghi—\verseline
		trattenga la memoria\verseline
		di uno sguardo aggrottato\verseline
		di una rinuncia feroce\verseline
		di una definitiva non-scelta
	\end{stanza}
\end{poem}

\clearpage

\poemtitle{hortus conclusus}

\begin{artItem}
	Edward Hopper, \begin{otherlanguage}{english}%
		Night Windows%
	\end{otherlanguage}
\end{artItem}

\begin{poem}
	\begin{stanza}
		che la mia casa sia\verseline
		non impedimento di muri\verseline
		non schermo al quartiere\verseline
		non spazio sottratto\verseline
		e nemmeno\verseline
		tana\verseline
		o torrione\verseline
		o faro\verseline
		o nascondiglio
	\end{stanza}

	\begin{stanza}
		voglio muri fatti di cotone\verseline
		di carta\verseline
		di zucchero filato\verseline
		di ragnatele\verseline
		o muri per niente
	\end{stanza}

	\begin{stanza}
		che la mia casa sia\verseline
		non filtro al respiro\verseline
		all'odore di fuori\verseline
		e nemmeno\verseline
		nasconda nel ventre\verseline
		vergognoso e sporco\verseline
		l'amore\verseline
		ma lo gridi ai passanti
	\end{stanza}

	\begin{stanza}
		nella notte del quartiere\verseline
		stanato da voglia di strade\verseline
		voglio sentirmi a casa\verseline
		più che dietro le tende\verseline
		della mia casa
	\end{stanza}
\end{poem}

\clearpage

\poemtitle{notti romane}

\begin{artItem}
	Renzo Vespignani, Periferia con gasometro
\end{artItem}

\begin{poem}
	\begin{stanza}
		notti di queste estate urbanizzata\verseline
		questa è la volta che vengo a stanarvi\verseline
		sui viali infilzati da fari stizziti\verseline
		corso Trieste l'Ostiense via Ugo Ojetti\verseline
		o meglio ancora sui ponti fuori mano\verseline
		sulle campate che scavalcano i binari\verseline
		cioè quel che fa la Serenissima al Collatino\verseline
		o meglio ancora agli angoli introversi\verseline
		lungo gli archi slabbrati del Raccordo\verseline
		o nell'ombra che fanno le fronde\verseline
		uscendo ribelli come i ricci d'un moretto\verseline
		come ad esempio al Mandrione e al Casaletto\verseline
		o lungo i recinti scuri delle ville urbane\verseline
		strette da grate inutili e invitanti\verseline
		e se mi scappa l'uzzolo\verseline
		tra i marmi e i travertini dei portoni\verseline
		che stanno intorno alle piazze sdegnose\verseline
		come la piazza Euclide o piazza\verseline
		di Buenos Aires
	\end{stanza}

	\begin{stanza}
		notti d'estate romana\verseline
		in fondo in fondo il dubbio mi resta\verseline
		se riesco a conoscervi come si deve\verseline
		ma si fa del nostro meglio\verseline
		e poi se non trovo voi\verseline
		non resto di sicuro a bocca asciutta\verseline
		nell'aria di fiori marciti e piombo caldo\verseline
		araldi prosseneti paraninfi?\verseline
		ne trovo quanti voglio\verseline
		chi mi vieta di amare in loro vostra specie?
	\end{stanza}
\end{poem}

\clearpage

\poemtitle{faul}

\begin{poem}
	\begin{stanza}
		notte autunnale\verseline
		prigioniera\verseline
		del luco più che\verseline
		delle civili mura
	\end{stanza}

	\begin{stanza}
		notte d'insonnia\verseline
		scandita in\verseline
		cicli infiniti\verseline
		di ascese e discese 
	\end{stanza}

	\begin{stanza}
		valle colma\verseline
		di vino drogato\verseline
		centro della mia\verseline
		gravità
	\end{stanza}
\end{poem}

\clearpage

\poemtitle{terra}

\begin{artItem}
	Kazimir Severinovič Malevič, \begin{otherlanguage}{russian}%
		Чёрный квадрат%
	\end{otherlanguage}
\end{artItem}

\begin{poem}
	\begin{stanza}
		per i nomi delle tue città\verseline
		che snocciolo dolcemente\verseline
		per le asprezze segrete\verseline
		dei tuoi cerreti antichi
	\end{stanza}

	\begin{stanza}
		dispiegata\verseline
		cartesiana\verseline
		splendidamente certa\verseline
		del tuo privilegio
	\end{stanza}

	\begin{stanza}
		tu irrevocabilmente\verseline
		mi appartieni
	\end{stanza}
\end{poem}
