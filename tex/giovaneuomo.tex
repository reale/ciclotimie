\begin{volumetitlepage}
	\volumetitle{Giovane uomo}
	\volumeheader{Giovane uomo}
\end{volumetitlepage}

\poemtitle{i}

\begin{poem}
	\begin{stanza}
		mi interrogano\verseline
		e io so solo intridermi\verseline
		di umidità fetente\verseline
		ogni fibra erta a nascondere\verseline
		il putrido di dentro\verseline
		degno di sprezzo
	\end{stanza}

	\begin{stanza}
		che non s’accorgano della paura\verseline
		che porto dentro\verseline
		solo questo mi empie la testa\verseline
		che non s’accorgano della paura\verseline
		che non so strapparmi da dentro
	\end{stanza}
\end{poem}

\clearpage

\poemtitle{ii}

\begin{poem}
	\begin{stanza}
		impalpabile come velo di sposa\verseline
		ovunque io mi volga\verseline
		qualcosa che non so definire\verseline
		intercetta la mia mano e il mio sguardo
	\end{stanza}

	\begin{stanza}
		e l'abbraccio virile d'un amico\verseline
		e la carezza dell'amante\verseline
		egualmente si spengono, quando mi raggiungono\verseline
		attraverso l'ignota membrana
	\end{stanza}

	\begin{stanza}
		come le parole, che affidate alla voce\verseline
		o alla penna inesperta\verseline
		non sanno gettare al vento\verseline
		la pesante zavorra, odiosa
	\end{stanza}

	\begin{stanza}
		ma destiamoci, e squarciamo\verseline
		il velo tenace, infecondo
	\end{stanza}

	\begin{stanza}
		che non resti inespresso il messaggio\verseline
		che non giunga sigillato ancora\verseline
		alle porte dell'Ade
	\end{stanza}

	\begin{stanza}
		che non trascorrano i giorni\verseline
		consumati interi nella finzione\verseline
		d'una ribellione tiepida
	\end{stanza}
\end{poem}

\clearpage

\poemtitle{iii}

\begin{artItem}
	Emilio Vedova, Città ostaggio
\end{artItem}

\begin{poem}
	\begin{stanza}
		che pensavi salendo all'orto\verseline
		maestro?
	\end{stanza}

	\begin{stanza}
		ti premevano il cuore le sorti dell'uomo?
	\end{stanza}

	\begin{stanza}
		non è che invece\verseline
		ti incantavano le luci della città\verseline
		il formicolio scuro di vita carnale?
	\end{stanza}

	\begin{stanza}
		che pensavi, maestro?
	\end{stanza}

	\begin{stanza}
		t'attendeva strazio... te ne curavi?\verseline
		prelibavi forse lo schiaffo del guitto\verseline
		la carne stracciata\verseline
		pregiato e raro esperire?\verseline
		o il repertorio intero dei lazzi\verseline
		dal terrore starnazzante\verseline
		all'indifferenza esibita con arte?
	\end{stanza}

	\begin{stanza}
		che pensavi salendo all'orto?\verseline
		barattare il resto dei giorni\verseline
		per un pugno di consapevolezza\verseline
		dura come concrezioni di perla in grembo di donna\verseline
		è da crepuscolari dei, maestro\verseline
		non da mortali
	\end{stanza}
\end{poem}

\clearpage

\poemtitle{iii}

\begin{poem}
	\begin{stanza}
		dormivano, diranno: dormivano\verseline
		ma io non dormo\verseline
		né dormono i compagni
	\end{stanza}

	\begin{stanza}
		soltanto che dolcezza\verseline
		tenere chiusi gli occhi\verseline
		le membra colte nella tunica\verseline
		strette alla terra
	\end{stanza}

	\begin{stanza}
		dentro gli occhi stanno ancora\verseline
		le piazze piene della capitale\verseline
		i fianchi pieni delle donne
	\end{stanza}

	\begin{stanza}
		lui non dorme\verseline
		tiene gli occhi aperti\verseline
		e le spalle piegate
	\end{stanza}

	\begin{stanza}
		è lontana da qui la mia casa
	\end{stanza}

	\begin{stanza}
		che dolcezza sapere che non si può cambiare il mondo\verseline
		che la vita ti tradisce \verseline
		è dolce anche la stanchezza della disfatta\verseline
		e noi sapevamo di essere esperti della vita\verseline
		più di lui\verseline
		che non guardava i fianchi delle donne
	\end{stanza}

	\begin{stanza}
		di lui\verseline
		che non dorme\verseline
		e tiene gli occhi aperti\verseline
		si sa\verseline
		e forse viene a cercarci
	\end{stanza}

	\begin{stanza}
		ma noi\verseline
		anche se tenevamo soltanto gli occhi chiusi\verseline
		un poco\verseline
		e il corpo stretto alla terra\verseline
		diranno che dormivamo
	\end{stanza}
\end{poem}

\clearpage

\poemtitle{iv}

\begin{poem}
	\begin{stanza}
		cosa vedesti di là dalle colonne?
	\end{stanza}

	\begin{stanza}
		una fuga infinita di mondi\verseline
		appena oltre il ciglio dell'abisso\verseline
		che forse ti sembrò bastare\verseline
		sporgere la mano per toccarli?
	\end{stanza}

	\begin{stanza}
		o forse soltanto ti balenò la paura\verseline
		indomata\verseline
		che mai abbandona le radici di ogni vita mortale?\verseline
		fu essa a sedurti?
	\end{stanza}

	\begin{stanza}
		andasti
	\end{stanza}
\end{poem}

\clearpage

\poemtitle{v}

\begin{artItem}
	Pablo Picasso, \begin{otherlanguage}{french}%
		La Dépouille du Minotaure en costume d’arlequin%
	\end{otherlanguage}
\end{artItem}

\begin{poem}
	\begin{stanza}
		lascia stare compagno\verseline
		il compagno caduto\verseline
		lascialo appeso\verseline
		ai corvi
	\end{stanza}

	\begin{stanza}
		portare a sua madre\verseline
		un corpo sconfitto\verseline
		con le parole mozzate\verseline
		o alla terra?
	\end{stanza}

	\begin{stanza}
		ora le tue braccia sono quelle\verseline
		di una vecchia prosciugata vergine zia\verseline
		che tiene il bimbetto di sua sorella\verseline
		braccia dure di legno verde
	\end{stanza}

	\begin{stanza}
		tu questa carne\verseline
		senza più parola\verseline
		la nascondi\verseline
		nella terra
	\end{stanza}

	\begin{stanza}
		tu non sai quanti viaggi\verseline
		ancora\verseline
		per tutti quegli altri\verseline
		compagni
	\end{stanza}
\end{poem}

\clearpage

\poemtitle{vi}

\begin{poem}
	\begin{stanza}
		non so che mi sgorga\verseline
		se sangue o acqua dal costato\verseline
		regolarmente trafitto\verseline
		o seme giovane dal sesso,\verseline
		né so che mi stira\verseline
		le membra antiche? no,\verseline
		ma tenere, e pegno di memoria
	\end{stanza}

	\begin{stanza}
		se mi insanguina la gola\verseline
		grido di carne offesa,\verseline
		o, a non saper dirlo,\verseline
		solo mi sforza la voce\verseline
		carne accesa di primavera;\verseline
		non so anche questo, e non so\verseline
		se siano in me lo stesso
	\end{stanza}
\end{poem}

\clearpage

\poemtitle{vii}

\begin{poem}
	\begin{stanza}
		ma farmi sapere al grido\verseline
		che non risponde al mio ancora\verseline
		di voci attese da tanto\verseline
		che gli hanno detto di lasciare\verseline
		a casa il loro odore di carne\verseline
		e non amano perciò che lacrime di cera\verseline
		e spregiano i miei succhi e gridano\verseline
		nello stormire insopportabile\verseline
		di cosa? forse ulivi nella notte
	\end{stanza}
\end{poem}

\clearpage

\poemtitle{viii}

\begin{artItem}
	Nunzio, Meteora
\end{artItem}

\begin{poem}
	\begin{stanza}
		comunque mi perdo nel canto\verseline
		di voci di carne comunque,\verseline
		di passione per dovere, ed è uguale\verseline
		offro, ne volete?, sangue pregiato\verseline
		da ornare le vostre labbra di carne\verseline
		e seme da stillarne sui vostri seni\verseline
		e ritrosia di puledro nascostamente storpio\verseline
		a divertire la vostra lucentezza\verseline
		e gemiti, pochi, che tolgo al padre
	\end{stanza}

	\begin{stanza}
		per me mi lascio un tremito solo mio
	\end{stanza}
\end{poem}
