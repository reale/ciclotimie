\chapter{Giovane uomo}

\poemtitle{i}

\begin{verse}
    mi interrogano\\
    e io so solo intridermi\\
    di umidità fetente\\
    ogni fibra erta a nascondere\\
    il putrido di dentro\\
    degno di sprezzo
\end{verse}

\begin{verse}
    che non s’accorgano della paura\\
    che porto dentro\\
    solo questo mi empie la testa\\
    che non s’accorgano della paura\\
    che non so strapparmi da dentro
\end{verse}

\clearpage

\poemtitle{ii}

\begin{verse}
    impalpabile come velo di sposa\\
    ovunque io mi volga\\
    qualcosa che non so definire\\
    intercetta la mia mano e il mio sguardo
\end{verse}

\begin{verse}
    e l'abbraccio virile d'un amico\\
    e la carezza dell'amante\\
    egualmente si spengono, quando mi raggiungono\\
    attraverso l'ignota membrana
\end{verse}

\begin{verse}
    come le parole, che affidate alla voce\\
    o alla penna inesperta\\
    non sanno gettare al vento\\
    la pesante zavorra, odiosa
\end{verse}

\begin{verse}
    ma destiamoci, e squarciamo\\
    il velo tenace, infecondo
\end{verse}

\begin{verse}
    che non resti inespresso il messaggio\\
    che non giunga sigillato ancora\\
    alle porte dell'Ade
\end{verse}

\begin{verse}
    che non trascorrano i giorni\\
    consumati interi nella finzione\\
    d'una ribellione tiepida
\end{verse}

\clearpage

\poemtitle{iii}

\begin{verse}
    che pensavi salendo all'orto\\
    maestro?
\end{verse}

\begin{verse}
    ti premevano il cuore le sorti dell'uomo?
\end{verse}

\begin{verse}
    non è che invece\\
    ti incantavano le luci della città\\
    il formicolio scuro di vita carnale?
\end{verse}

\begin{verse}
    che pensavi, maestro?
\end{verse}

\begin{verse}
    t'attendeva strazio... te ne curavi?\\
    prelibavi forse lo schiaffo del guitto\\
    la carne stracciata\\
    pregiato e raro esperire?\\
    o il repertorio intero dei lazzi\\
    dal terrore starnazzante\\
    all'indifferenza esibita con arte?
\end{verse}

\begin{verse}
    che pensavi salendo all'orto?\\
    barattare il resto dei giorni\\
    per un pugno di consapevolezza\\
    dura come concrezioni di perla in grembo di donna\\
    è da crepuscolari dei, maestro\\
    non da mortali
\end{verse}

\clearpage

\poemtitle{iii}

\begin{verse}
    dormivano, diranno: dormivano\\
    ma io non dormo\\
    né dormono i compagni
\end{verse}

\begin{verse}
    soltanto che dolcezza\\
    tenere chiusi gli occhi\\
    le membra colte nella tunica\\
    strette alla terra
\end{verse}

\begin{verse}
    dentro gli occhi stanno ancora\\
    le piazze piene della capitale\\
    i fianchi pieni delle donne
\end{verse}

\begin{verse}
    lui non dorme\\
    tiene gli occhi aperti\\
    e le spalle piegate
\end{verse}

\begin{verse}
    è lontana da qui la mia casa
\end{verse}

\begin{verse}
    che dolcezza sapere che non si può cambiare il mondo\\
    che la vita ti tradisce \\
    è dolce anche la stanchezza della disfatta\\
    e noi sapevamo di essere esperti della vita\\
    più di lui\\
    che non guardava i fianchi delle donne
\end{verse}

\begin{verse}
    di lui\\
    che non dorme\\
    e tiene gli occhi aperti\\
    si sa\\
    e forse viene a cercarci
\end{verse}

\begin{verse}
    ma noi\\
    anche se tenevamo soltanto gli occhi chiusi\\
    un poco\\
    e il corpo stretto alla terra\\
    diranno che dormivamo
\end{verse}

\clearpage

\poemtitle{iv}

\begin{verse}
    cosa vedesti di là dalle colonne?
\end{verse}

\begin{verse}
    una fuga infinita di mondi\\
    appena oltre il ciglio dell'abisso\\
    che forse ti sembrò bastare\\
    sporgere la mano per toccarli?
\end{verse}

\begin{verse}
    o forse soltanto ti balenò la paura\\
    indomata\\
    che mai abbandona le radici di ogni vita mortale?\\
    fu essa a sedurti?
\end{verse}

\begin{verse}
    andasti
\end{verse}

\clearpage

\poemtitle{v}

\begin{verse}
    lascia stare compagno\\
    il compagno caduto\\
    lascialo appeso\\
    ai corvi
\end{verse}

\begin{verse}
    portare a sua madre\\
    un corpo sconfitto\\
    con le parole mozzate\\
    o alla terra?
\end{verse}

\begin{verse}
    ora le tue braccia sono quelle\\
    di una vecchia prosciugata vergine zia\\
    che tiene il bimbetto di sua sorella\\
    braccia dure di legno verde
\end{verse}

\begin{verse}
    tu questa carne\\
    senza più parola\\
    la nascondi\\
    nella terra
\end{verse}

\begin{verse}
    tu non sai quanti viaggi\\
    ancora\\
    per tutti quegli altri\\
    compagni
\end{verse}

\clearpage

\poemtitle{vi}

\begin{verse}
    non so che mi sgorga\\
    se sangue o acqua dal costato\\
    regolarmente trafitto\\
    o seme giovane dal sesso,\\
    né so che mi stira\\
    le membra antiche? no,\\
    ma tenere, e pegno di memoria
\end{verse}

\begin{verse}
    se mi insanguina la gola\\
    grido di carne offesa,\\
    o, a non saper dirlo,\\
    solo mi sforza la voce\\
    carne accesa di primavera;\\
    non so anche questo, e non so\\
    se siano in me lo stesso
\end{verse}

\clearpage

\poemtitle{vii}

\begin{verse}
    ma farmi sapere al grido\\
    che non risponde al mio ancora\\
    di voci attese da tanto\\
    che gli hanno detto di lasciare\\
    a casa il loro odore di carne\\
    e non amano perciò che lacrime di cera\\
    e spregiano i miei succhi e gridano\\
    nello stormire insopportabile\\
    di cosa? forse ulivi nella notte
\end{verse}

\clearpage

\poemtitle{viii}

\begin{verse}
    comunque mi perdo nel canto\\
    di voci di carne comunque,\\
    di passione per dovere, ed è uguale\\
    offro, ne volete?, sangue pregiato\\
    da ornare le vostre labbra di carne\\
    e seme da stillarne sui vostri seni\\
    e ritrosia di puledro nascostamente storpio\\
    a divertire la vostra lucentezza\\
    e gemiti, pochi, che tolgo al padre
\end{verse}

\begin{verse}
    per me mi lascio un tremito solo mio
\end{verse}
