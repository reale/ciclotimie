\chapter{Estati}

\poemtitle{i}

\begin{verse}
    cielo velato di stracci\\
    e poi pioggia lenta\\
    e una spolverata di fresco\\
    che non ce la fa contro\\
    questa notte-di-già-estate
\end{verse}

\clearpage

\poemtitle{ii}

\begin{verse}
    non è più attesa\\
    presentimento\\
    veglia d'armi l'estate
\end{verse}

\begin{verse}
    è presenza\\
    stupefacente\\
    incontestabile presenza
\end{verse}

\clearpage

\poemtitle{iii}

\begin{verse}
    mi si avvolge attorno l'estate\\
    nemica ai resti d'ugge marzoline\\
    mi insidia ogni atto respiro pensiero
\end{verse}

\begin{verse}
    oggi è ancora colata di buon sole\\
    terra buona\\
    erba tagliata di fresco
\end{verse}

\begin{verse}
    ignota l'ansia di sfacelo\\
    il fiato pesante delle sere di luglio\\
    il presentimento dell'eterno fluire
\end{verse}

\clearpage

\poemtitle{iv}

\begin{verse}
    cominciare di luglio
\end{verse}

\begin{verse}
    andirivieni di certe sere\\
    il muoversi caldo e pieno\\
    di pedine sulla via del paese\\
    circondati ognuno di stupore
\end{verse}

\begin{verse}
    sta lontana ancora la resa\\
    di foglie dorate\\
    di foschie lievi e verso sera\\
    viali punteggiati dai lampioni
\end{verse}

\begin{verse}
    ora è solo lotta placata\\
    sospensione\\
    maturità tenera\\
    e sciogliersi lente le forze\\
    al riparo di muri a secco
\end{verse}

\clearpage

\poemtitle{v}

\begin{verse}
    conosco qualche giorno\\
    la voglia di riavvolgermi in me\\
    cercare coltri calde nella stagione\\
    e penombra viziata
\end{verse}

\begin{verse}
    conosco la sontuosità alata\\
    dell'estro al mattino\\
    appena scosso via il sonno\\
    poi chiedere un supplemento di oblio
\end{verse}

\begin{verse}
    conosco il magnetismo lento\\
    di risalire vicoli a memoria\\
    e canti in penombra e stupirsi\\
    di una scaglia di latta nel sole
\end{verse}

\begin{verse}
    conosco—che non se ne appanni\\
    l’incanto—albe precoci\\
    di una gioventù quasi intatta
\end{verse}

\begin{verse}
    ma intanto\\
    dell’estate\\
    ne faccio carne nuova
    pigramente
\end{verse}

\clearpage

\poemtitle{vi}

\begin{verse}
    a me\\
    non mi serve molto più di un salto\\
    imprevisto di sentiero tra l'erba\\
    danza d'ombre o sfrigolio di cicale
\end{verse}

\begin{verse}
    a me\\
    la sera mi basta poche volte\\
    posare il dorso su cemento di ferrovia\\
    caldo ancora di sole
\end{verse}

\begin{verse}
    a me\\
    credete mi ci voglia un'altra notte\\
    di carne colma e di sangue\\
    e di insondabilmente alieni fiati marini\\
    per acconsentire a questa estate?
\end{verse}

\clearpage

\poemtitle{vii}

\begin{verse}
    la dolcezza della veglia\\
    (quella colta con indugio colpevole)\\
    l'ho imparata appena le mattine\\
    di questa estate già accaduta
\end{verse}

\begin{verse}
    con le lenzuola umide sul capo\\
    gli occhi socchiusi\\
    in giro per una città dal porto quieto\\
    e poco oltre un oceano in una conca
\end{verse}

\clearpage

\poemtitle{viii}

\begin{verse}
    giornata nata tenera\\
    doppiato appena il meriggio\\
    fa già indovinare sua mollezza matura
\end{verse}

\clearpage

\poemtitle{ix}

\begin{verse}
    nel colmo della notte è più astuto il richiamo\\
    dell'altura lontana\\
    presto piena di ombre in questo punto dell'anno\\
    di altipiani cosparsi di covoni alla fine di luglio\\
    di poche case\\
    sola adiacenza tra due mondi
\end{verse}

\clearpage

\poemtitle{x}

\begin{verse}
    sonno che prende così all'improvviso\\
    e anche una voglia costruita ad arte\\
    ché tanto concedersi a sé si purifica\\
    nella luce integrale del mattino\\
    e neppure sembra improvvido esporsi\\
    allo sguardo furtivo di chi\\
    il caso vuole a sorprenderci\\
    abbracciati a noi stessi\\
    nella rete di poche lenzuola
\end{verse}

\clearpage

\poemtitle{xi}

\begin{verse}
    autunno non è temporale d'agosto\\
    autunno è presentimento di gelo\\
    che coglie il bagnante in un meriggio di sole
\end{verse}

\begin{verse}
    incongruo, quasi alieno, ma subito noto
\end{verse}

\clearpage

\poemtitle{xii}

\begin{verse}
    voglia di pomeriggi\\
    di quel miele torpido\\
    di odore di stoffa bagnata\\
    di legno buono che arde\\
    di ecatombi dorate sui viali\\
    di sere umide e mie
\end{verse}
