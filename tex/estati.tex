\begin{volumetitlepage}
	\volumetitle{Estati}
	\volumeheader{Estati}
\end{volumetitlepage}

\poemtitle{i}

\begin{poem}
	\begin{stanza}
		cielo velato di stracci\verseline
		e poi pioggia lenta\verseline
		e una spolverata di fresco\verseline
		che non ce la fa contro\verseline
		questa notte-di-già-estate
	\end{stanza}
\end{poem}

\clearpage

\poemtitle{ii}

\begin{poem}
	\begin{stanza}
		non è più attesa\verseline
		presentimento\verseline
		veglia d'armi l'estate
	\end{stanza}

	\begin{stanza}
		è presenza\verseline
		stupefacente\verseline
		incontestabile presenza
	\end{stanza}
\end{poem}

\clearpage

\poemtitle{iii}

\begin{poem}
	\begin{stanza}
		mi si avvolge attorno l'estate\verseline
		nemica ai resti d'ugge marzoline\verseline
		mi insidia ogni atto respiro pensiero
	\end{stanza}

	\begin{stanza}
		oggi è ancora colata di buon sole\verseline
		terra buona\verseline
		erba tagliata di fresco
	\end{stanza}

	\begin{stanza}
		ignota l'ansia di sfacelo\verseline
		il fiato pesante delle sere di luglio\verseline
		il presentimento dell'eterno fluire
	\end{stanza}
\end{poem}

\clearpage

\poemtitle{iv}

\begin{poem}
	\begin{stanza}
		cominciare di luglio
	\end{stanza}

	\begin{stanza}
		andirivieni di certe sere\verseline
		il muoversi caldo e pieno\verseline
		di pedine sulla via del paese\verseline
		circondati ognuno di stupore
	\end{stanza}

	\begin{stanza}
		sta lontana ancora la resa\verseline
		di foglie dorate\verseline
		di foschie lievi e verso sera\verseline
		viali punteggiati dai lampioni
	\end{stanza}

	\begin{stanza}
		ora è solo lotta placata\verseline
		sospensione\verseline
		maturità tenera\verseline
		e sciogliersi lente le forze\verseline
		al riparo di muri a secco
	\end{stanza}
\end{poem}

\clearpage

\poemtitle{v}

\begin{poem}
	\begin{stanza}
		conosco qualche giorno\verseline
		la voglia di riavvolgermi in me\verseline
		cercare coltri calde nella stagione\verseline
		e penombra viziata
	\end{stanza}

	\begin{stanza}
		conosco la sontuosità alata\verseline
		dell'estro al mattino\verseline
		appena scosso via il sonno\verseline
		poi chiedere un supplemento di oblio
	\end{stanza}

	\begin{stanza}
		conosco il magnetismo lento\verseline
		di risalire vicoli a memoria\verseline
		e canti in penombra e stupirsi\verseline
		di una scaglia di latta nel sole
	\end{stanza}

	\begin{stanza}
		conosco—che non se ne appanni\verseline
		l’incanto—albe precoci\verseline
		di una gioventù quasi intatta
	\end{stanza}

	\begin{stanza}
		ma intanto\verseline
		dell’estate\verseline
		ne faccio carne nuova
		pigramente
	\end{stanza}
\end{poem}

\clearpage

\poemtitle{vi}

\begin{artItem}
	Vasilij Vasil'evič Kandinskij, \begin{otherlanguage}{russian}%
		Цветные линии%
	\end{otherlanguage}
\end{artItem}

\begin{poem}
	\begin{stanza}
		a me\verseline
		non mi serve molto più di un salto\verseline
		imprevisto di sentiero tra l'erba\verseline
		danza d'ombre o sfrigolio di cicale
	\end{stanza}

	\begin{stanza}
		a me\verseline
		la sera mi basta poche volte\verseline
		posare il dorso su cemento di ferrovia\verseline
		caldo ancora di sole
	\end{stanza}

	\begin{stanza}
		a me\verseline
		credete mi ci voglia un'altra notte\verseline
		di carne colma e di sangue\verseline
		e di insondabilmente alieni fiati marini\verseline
		per acconsentire a questa estate?
	\end{stanza}
\end{poem}

\clearpage

\poemtitle{vii}

\begin{poem}
	\begin{stanza}
		la dolcezza della veglia\verseline
		(quella colta con indugio colpevole)\verseline
		l'ho imparata appena le mattine\verseline
		di questa estate già accaduta
	\end{stanza}

	\begin{stanza}
		con le lenzuola umide sul capo\verseline
		gli occhi socchiusi\verseline
		in giro per una città dal porto quieto\verseline
		e poco oltre un oceano in una conca
	\end{stanza}
\end{poem}

\clearpage

\poemtitle{viii}

\begin{poem}
	\begin{stanza}
		giornata nata tenera\verseline
		doppiato appena il meriggio\verseline
		fa già indovinare sua mollezza matura
	\end{stanza}
\end{poem}

\clearpage

\poemtitle{ix}

\begin{poem}
	\begin{stanza}
		nel colmo della notte è più astuto il richiamo\verseline
		dell'altura lontana\verseline
		presto piena di ombre in questo punto dell'anno\verseline
		di altipiani cosparsi di covoni alla fine di luglio\verseline
		di poche case\verseline
		sola adiacenza tra due mondi
	\end{stanza}
\end{poem}

\clearpage

\poemtitle{x}

\begin{poem}
	\begin{stanza}
		sonno che prende così all'improvviso\verseline
		e anche una voglia costruita ad arte\verseline
		ché tanto concedersi a sé si purifica\verseline
		nella luce integrale del mattino\verseline
		e neppure sembra improvvido esporsi\verseline
		allo sguardo furtivo di chi\verseline
		il caso vuole a sorprenderci\verseline
		abbracciati a noi stessi\verseline
		nella rete di poche lenzuola
	\end{stanza}
\end{poem}

\clearpage

\poemtitle{xi}

\begin{poem}
	\begin{stanza}
		autunno non è temporale d'agosto\verseline
		autunno è presentimento di gelo\verseline
		che coglie il bagnante in un meriggio di sole
	\end{stanza}

	\begin{stanza}
		incongruo, quasi alieno, ma subito noto
	\end{stanza}
\end{poem}

\clearpage

\poemtitle{xii}

\begin{poem}
	\begin{stanza}
		voglia di pomeriggi\verseline
		di quel miele torpido\verseline
		di odore di stoffa bagnata\verseline
		di legno buono che arde\verseline
		di ecatombi dorate sui viali\verseline
		di sere umide e mie
	\end{stanza}
\end{poem}
