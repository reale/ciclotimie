\chapter{Primavere}

\poemtitle{i}

\begin{verse}
    schiarisce il cielo\\
    un attimo prima fingeva inverno
\end{verse}

\clearpage

\poemtitle{ii}

\begin{verse}
    sonno mescolato di voci\\
    di luce\\
    di pioggia impròvvida di marzo
\end{verse}

\begin{verse}
    corsa a mezz’aria sui binari\\
    realtà o inganno dei sensi?
\end{verse}

\begin{verse}
    poi, giunto, ho scelto\\
    l’oblio delle coltri
\end{verse}

\clearpage

\poemtitle{iii}

\begin{otherlanguage}{french}
    \begin{verse}
        la haine et l’ivresse surveillent\\
        l’enfantement d’un jour
    \end{verse}
\end{otherlanguage}

\clearpage

\poemtitle{iv}

\begin{verse}
    si cammina come sospesi\\
    in un pulviscolo che si fa scuro\\
    e agguaglia terra e cielo
\end{verse}

\begin{verse}
    ci si aggrappa a questi pezzi\\
    di luce che invadono l’aria
\end{verse}

\clearpage

\poemtitle{v}

\begin{verse}
    il cielo è questo grembo immenso\\
    avvolgente e intoccabile\\
    sulle sue
\end{verse}

\begin{verse}
    con incongruità ben nota\\
    mi infiamma la primavera voglia di tana
\end{verse}

\clearpage

\poemtitle{vi}

\begin{verse}
    sere di marzo che già si slargano\\
    il cielo conca di barbagli\\
    una presenza-assenza\\
    di presentiti turbamenti marini
\end{verse}

\clearpage

\poemtitle{vii}

\begin{verse}
     primavera è anche questa\\
     col cielo pieno di pioggia\\
     perché si sente la lotta\\
     ma si sa la certezza del sereno
\end{verse}

\clearpage

\poemtitle{viii}

\begin{verse}
    ancora aspetto un apollo\\
    che venga a togliermi di dosso\\
    la pelle che mi germogliò\\
    l'anno passato
\end{verse}

\clearpage

\poemtitle{ix}

\begin{verse}
    questa primavera mi è scoppiata dentro\\
    come di consolidata prammatica\\
    ma poi che fine ha fatto la voglia
\end{verse}

\begin{verse}
    (che avevo)
\end{verse}

\begin{verse}
    di non lasciarne scorrere a vuoto\\
    i giorni gonfi di luce?
\end{verse}

\clearpage

\poemtitle{x}

\begin{verse}
    vento novello di maggio\\
    mi trascina fuori di casa\\
    mi vuole ebbro degli umori\\
    di un Mediterraneo ridesto...
\end{verse}
