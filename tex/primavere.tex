\chapter*{Primavere}

\poemtitle{i}

\begin{verse}
    che la notte di gennaio\\
    cuore di nebbia e strida di civette\\
    non sia inospite a pensieri meridiani\\
    che sono disumani\\
    come sappiamo
\end{verse}

\clearpage

\poemtitle{ii}

\begin{verse}
    la stagione già inclina\\
    al tempo che è più matura\\
    più piena la sua luce\\
    ma c’è rimasto l’inverno\\
    impigliato tra i gesti
\end{verse}

\clearpage

\poemtitle{iii}

\begin{verse}
    schiarisce il cielo\\
    un attimo prima fingeva inverno
\end{verse}

\clearpage

\poemtitle{iv}

\begin{verse}
    sonno mescolato di voci\\
    di luce\\
    di pioggia impròvvida di marzo
\end{verse}

\begin{verse}
    corsa a mezz’aria sui binari\\
    realtà o inganno dei sensi?
\end{verse}

\begin{verse}
    poi, giunto, ho scelto\\
    l’oblio delle coltri
\end{verse}

\clearpage

\poemtitle{v}

\begin{otherlanguage}{french}
    \begin{verse}
        la haine et l’ivresse surveillent\\
        l’enfantement d’un jour
    \end{verse}
\end{otherlanguage}

\clearpage

\poemtitle{vi}

\begin{verse}
    si cammina come sospesi\\
    in un pulviscolo che si fa scuro\\
    e agguaglia terra e cielo
\end{verse}

\begin{verse}
    ci si aggrappa a questi pezzi\\
    di luce che invadono l’aria
\end{verse}

\clearpage

\poemtitle{vii}

\begin{verse}
    il cielo è questo grembo immenso\\
    avvolgente e intoccabile\\
    sulle sue
\end{verse}

\begin{verse}
    con incongruità ben nota\\
    mi infiamma la primavera voglia di tana
\end{verse}

\clearpage

\poemtitle{viii}

\begin{verse}
    sere di marzo che già si slargano\\
    il cielo conca di barbagli\\
    una presenza-assenza\\
    di presentiti turbamenti marini
\end{verse}

\clearpage

\poemtitle{ix}

\begin{verse}
     primavera è anche questa\\
     col cielo pieno di pioggia\\
     perché si sente la lotta\\
     ma si sa la certezza del sereno
\end{verse}

\clearpage

\poemtitle{x}

\begin{verse}
    ancora aspetto chi\\
    che venga a togliermi di dosso\\
    la pelle che mi germogliò\\
    l'anno passato
\end{verse}

\clearpage

\poemtitle{xi}

\begin{verse}
    questa primavera mi è scoppiata dentro\\
    come di consolidata prammatica\\
    ma poi che fine ha fatto la voglia
\end{verse}

\begin{verse}
    (che avevo)
\end{verse}

\begin{verse}
    di non lasciarne scorrere a vuoto\\
    i giorni gonfi di luce?
\end{verse}
\chapter*{Pan}

\poemtitle{i}

\begin{verse}
    è specchio il dio\\
    è solitario amore\\
    è sguardo eternamente\\
    volto al fluire
\end{verse}

\begin{verse}
    ma sono la stessa cosa\\
    sonno veglia\\
    vigilanza incoscienza\\
    logos panico\\
    e frontiera\\
    e il grembo incerto\\
    della terra
\end{verse}

\clearpage

\poemtitle{ii}

\begin{verse}
    mi vieni incontro sul limite\\
    della sera o del sonno\\
    mi mostri permeabile\\
    la frontiera\\
    giacché ti si addice\\
    stare di fronte
\end{verse}

\begin{verse}
    pure sono tuoi per intero\\
    terrore del meriggio\\
    strada isolata\\
    folgorazione non mediata\\
    spazio senza ragione
\end{verse}

\clearpage

\poemtitle{iii}

\begin{verse}
    a cosa assomigliarti?
\end{verse}

\begin{verse}
    a una bestiola dei boschi\\
    un volpino o un leprotto che guizza tra l'erba\\
    o una daina giovinetta dalla tenera gola\\
    oppure a un agnello capriccioso\\
    ignaro ancora dei morsi?
\end{verse}

\begin{verse}
    ma sa l'agnello la rupe\\
    e la daina il signore radioso\\
    sanno il leprotto il volpino\\
    il silenzio al meriggio e i cani\\
    e l'inutile fuga?
\end{verse}
