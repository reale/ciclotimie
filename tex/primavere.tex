\chapter*{Primavere}

\poemtitle{i}

\begin{verse}
    che la notte di gennaio\\
    cuore di nebbia e strida di civette\\
    non sia inospite a pensieri meridiani\\
    che sono disumani\\
    come sappiamo
\end{verse}

\clearpage

\poemtitle{ii}

\begin{verse}
    la stagione già inclina\\
    al tempo che è più matura\\
    più piena la sua luce\\
    ma c'è rimasto l'inverno\\
    impigliato tra i gesti
\end{verse}

\clearpage

\poemtitle{iii}

\begin{verse}
    schiarisce il cielo\\
    un attimo prima fingeva inverno
\end{verse}

\clearpage

\poemtitle{iv}

\begin{verse}
    sonno mescolato di voci\\
    di luce\\
    di pioggia impròvvida di marzo
\end{verse}

\begin{verse}
    corsa a mezz'aria sui binari\\
    realtà o inganno dei sensi?
\end{verse}

\begin{verse}
    poi, giunto, ho scelto\\
    l'oblio delle coltri
\end{verse}

\clearpage

\poemtitle{v}

\begin{otherlanguage}{french}
    \begin{verse}
        la haine et l'ivresse surveillent\\
        l'enfantement d'un jour
    \end{verse}
\end{otherlanguage}

\clearpage

\poemtitle{vi}

\begin{verse}
    si cammina come sospesi\\
    in un pulviscolo che si fa scuro\\
    e agguaglia terra e cielo
\end{verse}

\begin{verse}
    ci si aggrappa a questi pezzi\\
    di luce che invadono l'aria
\end{verse}

\clearpage

\poemtitle{vii}

\begin{verse}
    il cielo è questo grembo immenso\\
    avvolgente e intoccabile\\
    sulle sue
\end{verse}

\begin{verse}
    con incongruità ben nota\\
    mi infiamma la primavera voglia di tana
\end{verse}

\clearpage

\poemtitle{viii}

\begin{verse}
    sere di marzo che già si slargano\\
    il cielo conca di barbagli\\
    una presenza-assenza\\
    di presentiti turbamenti marini
\end{verse}

\clearpage

\poemtitle{ix}

\begin{verse}
     primavera è anche questa\\
     col cielo pieno di pioggia\\
     perché si sente la lotta\\
     ma si sa la certezza del sereno
\end{verse}

\clearpage

\poemtitle{x}

\begin{verse}
    ancora aspetto chi\\
    venga a togliermi di dosso\\
    la pelle che mi germogliò\\
    l'anno passato
\end{verse}

\clearpage

\poemtitle{xi}

\begin{verse}
    questa primavera mi è scoppiata dentro\\
    come di consolidata prammatica\\
    ma poi che fine ha fatto la voglia
\end{verse}

\begin{verse}
    (che avevo)
\end{verse}

\begin{verse}
    di non lasciarne scorrere a vuoto\\
    i giorni gonfi di luce?
\end{verse}

\clearpage

\poemtitle{xii}

\begin{verse}
    ma sono la stessa cosa\\
    sonno veglia\\
    vigilanza incoscienza\\
    logos panico\\
    e frontiera\\
    e il grembo incerto\\
    della terra
\end{verse}

\clearpage

\poemtitle{xiii}

\begin{verse}
    mi vieni incontro sul limite\\
    della sera o del sonno\\
    mi mostri permeabile\\
    la frontiera\\
    giacché ti si addice\\
    stare di fronte
\end{verse}

\begin{verse}
    pure sono tuoi per intero\\
    terrore del meriggio\\
    strada isolata\\
    folgorazione non mediata\\
    spazio senza ragione\\
    sguardo eternamente volto al fluire
\end{verse}

\clearpage

\poemtitle{xiv}

\begin{verse}
    a cosa assomigliarti?
\end{verse}

\begin{verse}
    a una bestiola dei boschi\\
    un volpino o un leprotto che guizza tra l'erba\\
    o una daina giovinetta dalla tenera gola\\
    oppure a un agnello capriccioso\\
    ignaro ancora dei morsi?
\end{verse}

\begin{verse}
    ma sa l'agnello la rupe\\
    e la daina il signore radioso\\
    sanno il leprotto il volpino\\
    il silenzio al meriggio e i cani\\
    e l'inutile fuga?
\end{verse}

\clearpage

\poemtitle{xv}

\begin{verse}
    cielo velato di stracci\\
    e poi pioggia lenta\\
    e una spolverata di fresco\\
    che non ce la fa contro\\
    questa notte-di-già-estate
\end{verse}

\clearpage

\poemtitle{xvi}

\begin{verse}
    mi si avvolge attorno l'estate\\
    nemica ai resti d'ugge marzoline\\
    mi insidia ogni atto respiro pensiero
\end{verse}

\begin{verse}
    oggi è ancora colata di buon sole\\
    terra buona\\
    erba tagliata di fresco
\end{verse}

\begin{verse}
    ignota l'ansia di sfacelo\\
    il fiato pesante delle sere di luglio\\
    il presentimento dell'eterno fluire
\end{verse}

\clearpage

\poemtitle{xvii}

\begin{verse}
    cominciare di luglio
\end{verse}

\begin{verse}
    andirivieni di certe sere\\
    muoversi caldo e pieno\\
    di pedine sulla via del paese\\
    circondati ognuno di stupore
\end{verse}

\begin{verse}
    sta lontana ancora la resa\\
    di foglie dorate\\
    di foschie lievi e verso sera\\
    viali punteggiati dai lampioni
\end{verse}

\begin{verse}
    ora è solo lotta placata\\
    sospensione\\
    maturità tenera\\
    e sciogliersi lente le forze\\
    al riparo di muri a secco
\end{verse}

\clearpage

\poemtitle{xviii}

\begin{verse}
    conosco qualche giorno\\
    la voglia di riavvolgermi in me\\
    cercare coltri calde nella stagione\\
    e penombra viziata
\end{verse}

\begin{verse}
    conosco la sontuosità alata\\
    dell'estro al mattino\\
    appena scosso via il sonno\\
    poi chiedere un supplemento di oblio
\end{verse}

\begin{verse}
    conosco il magnetismo lento\\
    di risalire vicoli a memoria\\
    e canti in penombra e stupirsi\\
    di una scaglia di latta nel sole
\end{verse}

\begin{verse}
    conosco—che non se ne appanni\\
    l'incanto—albe precoci\\
    di una gioventù quasi intatta
\end{verse}

\begin{verse}
    ma intanto\\
    dell'estate\\
    ne faccio carne nuova
    pigramente
\end{verse}

\clearpage

\poemtitle{xix}

\begin{verse}
    a me\\
    non mi serve molto più di un salto\\
    imprevisto di sentiero tra l'erba\\
    danza d'ombre o sfrigolio di cicale
\end{verse}

\begin{verse}
    a me\\
    la sera mi basta poche volte\\
    posare il dorso su cemento di ferrovia\\
    caldo ancora di sole
\end{verse}

\begin{verse}
    a me\\
    credete mi ci voglia un'altra notte\\
    di carne colma e di sangue\\
    e di insondabilmente alieni fiati marini\\
    per acconsentire a questa estate?
\end{verse}

\clearpage

\poemtitle{xx}

\begin{verse}
    la dolcezza della veglia\\
    (quella colta con indugio colpevole)\\
    l'ho imparata appena le mattine\\
    di questa estate già accaduta
\end{verse}

\begin{verse}
    con le lenzuola umide sul capo\\
    gli occhi socchiusi\\
    in giro per una città dal porto quieto\\
    e poco oltre un oceano in una conca
\end{verse}

\clearpage

\poemtitle{xxi}

\begin{verse}
    giornata nata tenera\\
    doppiato appena il meriggio\\
    fa già indovinare sua mollezza matura
\end{verse}

\clearpage

\poemtitle{xxii}

\begin{verse}
    nel colmo della notte è più astuto il richiamo\\
    dell'altura lontana\\
    presto piena di ombre in questo punto dell'anno\\
    di altipiani cosparsi di covoni alla fine di luglio\\
    di poche case\\
    sola adiacenza tra due mondi
\end{verse}

\clearpage

\poemtitle{xxiii}

\begin{verse}
    sonno che prende così all'improvviso\\
    e anche una voglia costruita ad arte\\
    ché tanto concedersi a sé si purifica\\
    nella luce integrale del mattino\\
    e neppure sembra improvvido esporsi\\
    allo sguardo furtivo di chi\\
    il caso vuole a sorprenderci\\
    abbracciati a noi stessi\\
    nella rete di poche lenzuola
\end{verse}

\clearpage

\poemtitle{xxiv}

\begin{verse}
    voglia di pomeriggi\\
    di quel miele torpido\\
    di odore di stoffa bagnata\\
    di legno buono che arde\\
    di ecatombi dorate sui viali\\
    di sere umide e mie
\end{verse}
