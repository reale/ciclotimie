\chapter*{Kresy Wschodnie}

\poemtitle{boemia}

\begin{verse}
    da dove viene la voglia\\
    di brume e di brine\\
    di acque dimora alle streghe\\
    di boschi orlati di neve\\
    di bacche vermiglie tra i rovi\\
    di giorni brevissimi\\
    di guglie di umor nero\\
    di cianfrusaglie polverose\\
    di gabbiani sul fiume\\
    di inconsapevoli cementi\\
    di maestri di pietra\\
    di vocali lunghe e brevi\\
    di seducenti problematiche elle
\end{verse}

\begin{verse}
    da dove viene la voglia\\
    di una casa che non è di mio padre\\
    di una lingua che i miei amici sanno straniera\\
    di ali di cicogna che forse mi sfiorarono\\
    tornando al principio di primavera\\
    agli spioventi umidi ancora di neve
\end{verse}

\clearpage

\poemtitle{krakonoš}

\begin{verse}
    io ti conosco
\end{verse}

\begin{verse}
    sei la solitudine aspra\\
    sei l’andare interminabile\\
    sei il perdersi tra le nevi
\end{verse}

\begin{verse}
    accigliata vergine ferocia
\end{verse}

\begin{verse}
    quelli che non ti sanno\\
    a che gli serve cantare le altre pietre\\
    quelle pietre coltivate di piccoli quartieri?
\end{verse}

\begin{verse}
    di laggiù pochi conoscono\\
    il tuo apparire nella neve\\
    ma nessuno come me sa dire\\
    nel confine aperto dell’orizzonte\\
    dov’è che s’innalzano\\
    le cento guglie
\end{verse}

\clearpage

\poemtitle{strade}

\begin{verse}
    un'appropriazione inscritta\\
    nel gioco dei nomi\\
    imparati questi\\
    quelli saputi da prima\\
    altri ancora ignoti
\end{verse}

\begin{verse}
    senso smarrito e ritrovato\\
    tra i percorsi tra le strade\\
    nella tela di binari e vecchie corti\\
    botteghe opifici falansteri mulini\\
    torrenti dal corso sigillato
\end{verse}

\begin{verse}
    vie strutte dal lungo andare\\
    e libertà conosciuta appena\\
    nel dipanare l'infinita rete\\
    nomi a milioni di milioni\\
    storia d'ogni pietra di selciato
\end{verse}

\begin{verse}
    i nomi trascorrono l'uno nell'altro\\
    ed uno basta a richiamare l'incanto\\
    d'un cielo fosco di novembre\\
    di gabbiani sedotti dalle arcate\\
    di tram e lampioni e vibrante andare
\end{verse}

\begin{verse}
    mie strade miei nomi lontani\\
    recalcitranti alla memoria\\
    recalcitranti all'oblio\\
    perché vi accrescete ogni ora\\
    di sensi nuovi e rimandi segreti?
\end{verse}

\begin{verse}
    perché tanto mi inasprite lo strazio\\
    dell'assenza e dell'attesa?
\end{verse}

\clearpage

\poemtitle{výtoň}

\begin{verse}
    già spettatore alle ibridazioni\\
    di ninfe dispettose con il fiume\\
    cosa mi fai oggi?\\
    ti cimenti in pregio d'antico lignaggio con la rocca sovrana?\\
    lei vanta i fasti della scaturigine ancestrale...
\end{verse}

\begin{verse}
    tu?
\end{verse}

\begin{verse}
    tu\\
    piccolo cubo remoto\\
    lasci che la venustà dei tuoi secoli\\
    si sdegni tra le cortine di bosco posticcio\\
    tra l’anticaglia di guglie e piazze\\
    esibite senza pudore
\end{verse}

\begin{verse}
    ma non sai che lungi da te non più di tre volte\\
    cento passi\\
    corre il turista smemorato\\
    guizza il ruglio d'acciaio del tram?
\end{verse}

\clearpage

\poemtitle{nuselský most}

\begin{verse}
    dall'alto di quarantadue metri\\
    in cemento precompresso\\
    lascio cadere lo sguardo\\
    un rado formicaio di vita\\
    uno laggiù osa alzare lo sguardo\\
    non mi vede\\
    non vede accanto a me\\
    un milione di macchine al mese\\
    non vede credo neppure intero il sole\\
    ne ruba un bello spicchio il ponte
\end{verse}

\clearpage

\poemtitle{petřín}

\begin{verse}
    l'inizio della collina\\
    laggiù tra le case\\
    è mistero assoluto\\
    dalla parte di Malá Strana\\
    si perde Praga\\
    si smarrisce\\
    cela lo sgomento in gesti di pietra\\
    ma troppo presto\\
    irrompe il vibrare del bosco\\
    troppo rapido troppo subdolo\\
    il trapasso\\
    per questo dicono siano un po' strambi\\
    quelli di Malá Strana
\end{verse}

\clearpage

\poemtitle{náměstí míru}

\begin{verse}
    minuscola gloria stretta\\
    tra asburgici fasti\\
    è tuo il fascino che non si svela\\
    il segreto di quartieri ulteriori\\
    ambigui e incipriati\\
    come vecchie cortigiane
\end{verse}

\clearpage

\poemtitle{botič}

\begin{verse}
    nasci virgulto della terra ceca\\
    ma non ti si conoscono esitazioni\\
    sai subito dove andare\\
    e passi deciso campi e paesi e campagne\\
    Čenětice Olešky Průhonice\\
    non sembri pago dei nomi presàghi \\
    testardo tenti altri nomi\\
    campagne e campi e case\\
    non più villaggi ormai ma già sobborghi\\
    Křeslice\\
    che ti fa suo orgoglio e t'intitola un parco\\
    e poi mulini antichi e rapide strade\\
    (i mulini parlerebbero ancora di te)\\
    accogli vergine il sapore di altre acque\\
    e ora quartieri grandi\\
    Hostivař Záběhlice\\
    la strada ferrata? ti fa il verso per un tratto\\
    e hai un sogno\\
    allarghi le tue spire\\
    docile insinuante e quando piove impetuoso\\
    Michle\\
    Nusle\\
    è a un passo l'ombra sacra di Vyšehrad\\
    ecco ti corrono al fianco i tram e già\\
    un turista ti osserva stupito\\
    che non sia uno soltanto il fiume di Praga\\
    ma poco ti fu concesso ancora di vita\\
    da geografia e piani regolatori\\
    ti succhia un antro\\
    e qualcuno dirà che muori senza gloria\\
    fai l'ultimo miglio alla cieca\\
    emergi alla luce soltanto un istante\\
    all'ombra del vecchio ponte di ferro\\
    t'accoglie vorace la Vltava\\
    ma lascialo stare il grande fiume \\
    gloriati d'essere primo tra pari\\
    e primo, senza pari, per me
\end{verse}

\clearpage

\poemtitle{baltico}

\begin{verse}
    ho comprato due cestini\\
    di maliny, da una vecchia,\\
    l’uno per cinque złoty
\end{verse}

\begin{verse}
    ho comprato due cestini\\
    simili a due mele granate\\
    piene di chicchi ferrigni
\end{verse}

\begin{verse}
    laggiù, accanto al mercato\\
    coperto, fiammante di nuovi mattoni rossi\\
    nell'ulica Pańska
\end{verse}

\begin{verse}
    ed eccomi\\
    con le mani screziate di carminio\\
    a succhiare i piccoli frutti\\
    tenue sapore
\end{verse}

\clearpage

\poemtitle{nowa huta}

\begin{verse}
    da amare avete cielo d'acciaio\\
    terra dura a rompersi\\
    dal vomere nella stagione\\
    che il sole è scarsa moneta
\end{verse}

\begin{verse}
    dimenticate le opere antiche\\
    la terra seminatela a cemento\\
    il cielo bucatelo con l'ossatura\\
    di quartieri sorti in una notte
\end{verse}

\begin{verse}
    lasciatele accrescersi le scaglie\\
    dure del leviatano le pareti sottili\\
    messe ordinata docile di cubicoli\\
    promessa di nuova abbondanza
\end{verse}

\begin{verse}
    date man forte agli operai\\
    che attendono al gran lavoro\\
    in ben regolati turni\\
    di breve intervallo
\end{verse}

\begin{verse}
    guardate lo scheletro\\
    coprirsi di scaglie dure\\
    nelle notti d'inverno\\
    fatte giorno dai fari
\end{verse}

\begin{verse}
    imparate ad amarli questi\\
    parallelepipedi dalle sottili pareti\\
    alcova e focolare e culla alla prole\\
    orologeria del buon ordine operaio
\end{verse}

\clearpage

\poemtitle{varsavia}

\begin{verse}
    grattacieli incurvati\\
    da orgoglio virile\\
    non sanno più guardare in alto\\
    senza tendersi in arco ardente di uomo
\end{verse}

\clearpage

\poemtitle{città danubiana}

\begin{verse}
    viali astuti s'avvolgono\\
    tra le case prima di inerpicarsi\\
    prima di dire al viandante\\
    che non troppo lo salva\\
    l’andare in giro
\end{verse}

\begin{verse}
    ma la città non è che una frattura\\
    non le si chiede che di raccogliere\\
    i notturnali sogni\\
    trionfi di combinatorica incestuosa
\end{verse}

\clearpage

\poemtitle{albania}

\begin{verse}
    da schiavi lo sguardo dei maschi\\
    ma da schiavi non domati\\
    gonfi di sfida obliqua
\end{verse}

\begin{verse}
    da regine lo sguardo delle donne\\
    cui l'ombra appena d'un sorriso\\
    è già di troppo a suscitare\\
    lo straniero tronfio e inetto\\
    e un gesto basta a impetrarlo
\end{verse}

\clearpage

\poemtitle{atene}

\begin{verse}
    voi marmi antichi saldi e severi\\
    indifferenti all'ammirazione\\
    o indifferenti e basta\\
    ammutoliti superstiti assediati
\end{verse}

\begin{verse}
    voi marmi vorrete perdonarmi\\
    se profittando dell'elevazione della rupe\\
    io per un attimo vi do le spalle\\
    ché cerco cosa mi chiama qui: forse
\end{verse}

\begin{verse}
    un tappeto di scatole bianche nel sole\\
    più abbondanti dei grani di sabbia\\
    di tutte le isole messe insieme\\
    non una sola compagna a un'altra: o forse
\end{verse}

\begin{verse}
    sotto i piedi la roccia cotta e dura\\
    sorella a quella degli scogli\\
    e laggiù il mare — un poco più vecchio di voi\\
    ma un poco meno segnato dagli anni
\end{verse}

\clearpage

\poemtitle{taranto}

\begin{verse}
    enfia inquieta\\
    litania di piaghe\\
    lavate di mare\\
    fragranti di sangue\\
    asperse di sale
\end{verse}

\begin{verse}
    di fere quel sanguine\\
    di mostri pelagici\\
    mogliere lubriche\\
    o di carni stillanti\\
    del corpo del nume
\end{verse}

\begin{verse}
    ecce homo\\
    ma della terra tradita\\
    chi sa le ulcere?
\end{verse}

\begin{verse}
    e il sole appannato\\
    e l’aria che assorbe il morbo\\
    chi li dice?
\end{verse}
