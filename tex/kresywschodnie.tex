\begin{volumetitlepage}
	\volumetitle{Kresy Wschodnie}
	\volumeheader{Kresy Wschodnie}
\end{volumetitlepage}

\poemtitle{boemia}

\begin{artItem}
	Egon Schiele, \begin{otherlanguage}{german}%
		Vier Bäume%
	\end{otherlanguage}
\end{artItem}

\begin{poem}
	\begin{stanza}
		da dove viene la voglia\verseline
		di brume e di brine\verseline
		di acque dimora alle streghe\verseline
		di boschi orlati di neve\verseline
		di bacche vermiglie tra i rovi\verseline
		di giorni brevissimi\verseline
		di guglie di umor nero\verseline
		di cianfrusaglie polverose\verseline
		di gabbiani sul fiume\verseline
		di inconsapevoli cementi\verseline
		di maestri di pietra\verseline
		di vocali lunghe e brevi\verseline
		di seducenti problematiche elle
	\end{stanza}

	\begin{stanza}
		da dove viene la voglia\verseline
		di una casa che non è di mio padre\verseline
		di una lingua che i miei amici sanno straniera\verseline
		di ali di cicogna che forse mi sfiorarono\verseline
		tornando al principio di primavera\verseline
		agli spioventi umidi ancora di neve
	\end{stanza}
\end{poem}

\clearpage

\poemtitle{krakonoš}

\begin{poem}
	\begin{stanza}
		io ti conosco
	\end{stanza}

	\begin{stanza}
		sei la solitudine aspra\verseline
		sei l’andare interminabile\verseline
		sei il perdersi tra le nevi
	\end{stanza}

	\begin{stanza}
		accigliata vergine ferocia
	\end{stanza}

	\begin{stanza}
		quelli che non ti sanno\verseline
		a che gli serve cantare le altre pietre\verseline
		quelle pietre coltivate di piccoli quartieri?
	\end{stanza}

	\begin{stanza}
		di laggiù pochi conoscono\verseline
		il tuo apparire nella neve\verseline
		ma nessuno come me sa dire\verseline
		nel confine aperto dell’orizzonte\verseline
		dov’è che s’innalzano\verseline
		le cento guglie
	\end{stanza}
\end{poem}

\clearpage

\poemtitle{strade}

\begin{artItem}
	Josef Sudek, \begin{otherlanguage}{german}%
		Moldau mit Nationaltheater%
	\end{otherlanguage}
\end{artItem}

\begin{poem}
	\begin{stanza}
		un'appropriazione inscritta\verseline
		nel gioco dei nomi\verseline
		imparati questi\verseline
		quelli saputi da prima\verseline
		altri ancora ignoti
	\end{stanza}

	\begin{stanza}
		senso smarrito e ritrovato\verseline
		tra i percorsi tra le strade\verseline
		nella tela di binari e vecchie corti\verseline
		botteghe opifici falansteri mulini\verseline
		torrenti dal corso sigillato
	\end{stanza}

	\begin{stanza}
		vie strutte dal lungo andare\verseline
		e libertà conosciuta appena\verseline
		nel dipanare l'infinita rete\verseline
		nomi a milioni di milioni\verseline
		storia d'ogni pietra di selciato
	\end{stanza}

	\begin{stanza}
		i nomi trascorrono l'uno nell'altro\verseline
		ed uno basta a richiamare l'incanto\verseline
		d'un cielo fosco di novembre\verseline
		di gabbiani sedotti dalle arcate\verseline
		di tram e lampioni e vibrante andare
	\end{stanza}

	\begin{stanza}
		mie strade miei nomi lontani\verseline
		recalcitranti alla memoria\verseline
		recalcitranti all'oblio\verseline
		perché vi accrescete ogni ora\verseline
		di sensi nuovi e rimandi segreti?
	\end{stanza}

	\begin{stanza}
		perché tanto mi inasprite lo strazio\verseline
		dell'assenza e dell'attesa?
	\end{stanza}
\end{poem}

\clearpage

\poemtitle{výtoň}

\begin{poem}
	\begin{stanza}
		già spettatore alle ibridazioni\verseline
		di ninfe dispettose con il fiume\verseline
		cosa mi fai oggi?\verseline
		ti cimenti in pregio d'antico lignaggio con la rocca sovrana?\verseline
		lei vanta i fasti della scaturigine ancestrale...
	\end{stanza}

	\begin{stanza}
		tu?
	\end{stanza}

	\begin{stanza}
		tu\verseline
		piccolo cubo remoto\verseline
		lasci che la venustà dei tuoi secoli\verseline
		si sdegni tra le cortine di bosco posticcio\verseline
		tra l’anticaglia di guglie e piazze\verseline
		esibite senza pudore
	\end{stanza}

	\begin{stanza}
		ma non sai che lungi da te non più di tre volte\verseline
		cento passi\verseline
		corre il turista smemorato\verseline
		guizza il ruglio d'acciaio del tram?
	\end{stanza}
\end{poem}

\clearpage

\poemtitle{nuselský most}

\begin{poem}
	\begin{stanza}
		dall'alto di quarantadue metri\verseline
		in cemento precompresso\verseline
		lascio cadere lo sguardo\verseline
		un rado formicaio di vita\verseline
		uno laggiù osa alzare lo sguardo\verseline
		non mi vede\verseline
		non vede accanto a me\verseline
		un milione di macchine al mese\verseline
		non vede credo neppure intero il sole\verseline
		ne ruba un bello spicchio il ponte
	\end{stanza}
\end{poem}

\clearpage

\poemtitle{petřín}

\begin{artItem}
	Stanislav Feikl, \begin{otherlanguage}{czech}%
		Hradčany z Malé Strany%
	\end{otherlanguage}
\end{artItem}

\begin{poem}
	\begin{stanza}
		l'inizio della collina\verseline
		laggiù tra le case\verseline
		è mistero assoluto\verseline
		dalla parte di Malá Strana\verseline
		si perde Praga\verseline
		si smarrisce\verseline
		cela lo sgomento in gesti di pietra\verseline
		ma troppo presto\verseline
		irrompe il vibrare del bosco\verseline
		troppo rapido troppo subdolo\verseline
		il trapasso\verseline
		per questo dicono siano un po' strambi\verseline
		quelli di Malá Strana
	\end{stanza}
\end{poem}

\clearpage

\poemtitle{náměstí míru}

\begin{poem}
	\begin{stanza}
		minuscola gloria stretta\verseline
		tra asburgici fasti\verseline
		è tuo il fascino che non si svela\verseline
		il segreto di quartieri ulteriori\verseline
		ambigui e incipriati\verseline
		come vecchie cortigiane
	\end{stanza}
\end{poem}

\clearpage

\poemtitle{botič}

\begin{poem}
	\begin{stanza}
		nasci virgulto della terra ceca\verseline
		ma non ti si conoscono esitazioni\verseline
		sai subito dove andare\verseline
		e passi deciso campi e paesi e campagne\verseline
		Čenětice Olešky Průhonice\verseline
		non sembri pago dei nomi presàghi \verseline
		testardo tenti altri nomi\verseline
		campagne e campi e case\verseline
		non più villaggi ormai ma già sobborghi\verseline
		Křeslice\verseline
		che ti fa suo orgoglio e t'intitola un parco\verseline
		e poi mulini antichi e rapide strade\verseline
		(i mulini parlerebbero ancora di te)\verseline
		accogli vergine il sapore di altre acque\verseline
		e ora quartieri grandi\verseline
		Hostivař Záběhlice\verseline
		la strada ferrata? ti fa il verso per un tratto\verseline
		e hai un sogno\verseline
		allarghi le tue spire\verseline
		docile insinuante e quando piove impetuoso\verseline
		Michle\verseline
		Nusle\verseline
		è a un passo l'ombra sacra di Vyšehrad\verseline
		ecco ti corrono al fianco i tram e già\verseline
		un turista ti osserva stupito\verseline
		che non sia uno soltanto il fiume di Praga\verseline
		ma poco ti fu concesso ancora di vita\verseline
		da geografia e piani regolatori\verseline
		ti succhia un antro\verseline
		e qualcuno dirà che muori senza gloria\verseline
		fai l'ultimo miglio alla cieca\verseline
		emergi alla luce soltanto un istante\verseline
		all'ombra del vecchio ponte di ferro\verseline
		t'accoglie vorace la Vltava\verseline
		ma lascialo stare il grande fiume \verseline
		gloriati d'essere primo tra pari\verseline
		e primo, senza pari, per me
	\end{stanza}
\end{poem}

\clearpage

\poemtitle{baltico}

\begin{poem}
	\begin{stanza}
		ho comprato due cestini\verseline
		di maliny, da una vecchia,\verseline
		l’uno per cinque złoty
	\end{stanza}

	\begin{stanza}
		ho comprato due cestini\verseline
		simili a due mele granate\verseline
		piene di chicchi ferrigni
	\end{stanza}

	\begin{stanza}
		laggiù, accanto al mercato\verseline
		coperto, fiammante di nuovi mattoni rossi\verseline
		nell'ulica Pańska
	\end{stanza}

	\begin{stanza}
		ed eccomi\verseline
		con le mani screziate di carminio\verseline
		a succhiare i piccoli frutti\verseline
		tenue sapore
	\end{stanza}
\end{poem}

\clearpage

\poemtitle{nowa huta}

\begin{poem}
	\begin{stanza}
		da amare avete cielo d'acciaio\verseline
		terra dura a rompersi\verseline
		dal vomere nella stagione\verseline
		che il sole è scarsa moneta
	\end{stanza}

	\begin{stanza}
		dimenticate le opere antiche\verseline
		la terra seminatela a cemento\verseline
		il cielo bucatelo con l'ossatura\verseline
		di quartieri sorti in una notte
	\end{stanza}

	\begin{stanza}
		lasciatele accrescersi le scaglie\verseline
		dure del leviatano le pareti sottili\verseline
		messe ordinata docile di cubicoli\verseline
		promessa di nuova abbondanza
	\end{stanza}

	\begin{stanza}
		date man forte agli operai\verseline
		che attendono al gran lavoro\verseline
		in ben regolati turni\verseline
		di breve intervallo
	\end{stanza}

	\begin{stanza}
		guardate lo scheletro\verseline
		coprirsi di scaglie dure\verseline
		nelle notti d'inverno\verseline
		fatte giorno dai fari
	\end{stanza}

	\begin{stanza}
		imparate ad amarli questi\verseline
		parallelepipedi dalle sottili pareti\verseline
		alcova e focolare e culla alla prole\verseline
		orologeria del buon ordine operaio
	\end{stanza}
\end{poem}

\clearpage

\poemtitle{varsavia}

\begin{poem}
	\begin{stanza}
		grattacieli incurvati\verseline
		da orgoglio virile\verseline
		non sanno più guardare in alto\verseline
		senza tendersi in arco ardente di uomo
	\end{stanza}
\end{poem}

\clearpage

\poemtitle{città danubiana}

\begin{poem}
	\begin{stanza}
		viali astuti s'avvolgono\verseline
		tra le case prima di inerpicarsi\verseline
		prima di dire al viandante\verseline
		che non troppo lo salva\verseline
		l’andare in giro
	\end{stanza}

	\begin{stanza}
		ma la città non è che una frattura\verseline
		non le si chiede che di raccogliere\verseline
		i notturnali sogni\verseline
		trionfi di combinatorica incestuosa
	\end{stanza}
\end{poem}

\clearpage

\poemtitle{albania}

\begin{poem}
	\begin{stanza}
		da schiavi lo sguardo dei maschi\verseline
		ma da schiavi non domati\verseline
		gonfi di sfida obliqua
	\end{stanza}

	\begin{stanza}
		da regine lo sguardo delle donne\verseline
		cui l'ombra appena d'un sorriso\verseline
		è già di troppo a suscitare\verseline
		lo straniero tronfio e inetto\verseline
		e un gesto basta a impetrarlo
	\end{stanza}
\end{poem}

\clearpage

\poemtitle{atene}

\begin{artItem}
	Yiannis Moralis, \begin{otherlanguage}{greek}%
		Σχόλιο για τη συλλογή «Ποιήματα» του Γιώργου Σεφέρη%
	\end{otherlanguage}
\end{artItem}

\begin{poem}
	\begin{stanza}
		voi marmi antichi saldi e severi\verseline
		indifferenti all'ammirazione\verseline
		o indifferenti e basta\verseline
		ammutoliti superstiti assediati
	\end{stanza}

	\begin{stanza}
		voi marmi vorrete perdonarmi\verseline
		se profittando dell'elevazione della rupe\verseline
		io per un attimo vi do le spalle\verseline
		ché cerco cosa mi chiama qui: forse
	\end{stanza}

	\begin{stanza}
		un tappeto di scatole bianche nel sole\verseline
		più abbondanti dei grani di sabbia\verseline
		di tutte le isole messe insieme\verseline
		non una sola compagna a un'altra: o forse
	\end{stanza}

	\begin{stanza}
		sotto i piedi la roccia cotta e dura\verseline
		sorella a quella degli scogli\verseline
		e laggiù il mare — un poco più vecchio di voi\verseline
		ma un poco meno segnato dagli anni
	\end{stanza}
\end{poem}

\clearpage

\poemtitle{taranto}

\begin{artItem}
	Mimmo Paladino, Veronica
\end{artItem}

\begin{poem}
	\begin{stanza}
		enfia inquieta\verseline
		litania di piaghe\verseline
		lavate di mare\verseline
		fragranti di sangue\verseline
		asperse di sale
	\end{stanza}

	\begin{stanza}
		di fere quel sanguine\verseline
		di mostri pelagici\verseline
		mogliere lubriche\verseline
		o di carni stillanti\verseline
		del corpo del nume
	\end{stanza}

	\begin{stanza}
		ecce homo\verseline
		ma della terra tradita\verseline
		chi sa le ulcere?
	\end{stanza}

	\begin{stanza}
		e il sole appannato\verseline
		e l’aria che assorbe il morbo\verseline
		chi li dice?
	\end{stanza}
\end{poem}
