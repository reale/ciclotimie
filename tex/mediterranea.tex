\poemtitle{parodo}

	\begin{verse}
		Maestro
	\end{verse}

	\begin{verse}
		mi imponi ritorno\\
		alle coste scorte quegli anni e forse\\
		scordate
	\end{verse}

	\begin{verse}
		gridi richiami
	\end{verse}

	\begin{verse}
		ma fanciullezza è passata\\
		e che vale se fosse ignara\\
		meno di questi giorni maturi\\
		meno di questa efficienza\\
		fatta di cosa?
	\end{verse}

	\begin{verse}
		non sai se magari nulla più io so\\
		di allora\\
		se non voglia di ritorno
	\end{verse}

	\begin{verse}
		ma Itaca è fuori portata
	\end{verse}

	\begin{verse}
		allora a che gridi richiami? ancora?
	\end{verse}

	\begin{verse}
		di quanto sperammo\\
		di quanto credemmo\\
		lasciai andare la rotta\\
		né mi resta ricordo\\
		se non di spechi senz’occhi\\
		aperti sull’erebo
	\end{verse}

	\begin{verse}
		e tu ancora\\
		urgi il ritorno\\
		alle spiagge remote\\
		agli scogli\\
		alla spuma fecondata\\
		e irragionevole
	\end{verse}

	\begin{verse}
		è lontana da qui Itaca
	\end{verse}

	\begin{verse}
		eppure
	\end{verse}


\clearpage

\poemtitle{mare i}


	\begin{verse}
		crespo azzurro teso\\
		panno steso d’azzurro\\
		alle bande d’acciaio
	\end{verse}

	\begin{verse}
		supremazia d’un elemento\\
		spiegato\\
		tra lembi sterili d’un altro
	\end{verse}

	\begin{verse}
		ci sia feconda\\
		la tua collera
	\end{verse}

	\begin{verse}
		ci sia magistero\\
		il metro nascosto\\
		dei tuoi capricci
	\end{verse}

	\begin{verse}
		ci sia fondamento\\
		di tutti i giorni\\
		attingere in te\\
		scaturigine e termine
	\end{verse}

	\begin{verse}
		ci sia dato
		per sapienza estrema\\
		di riconoscere\\
		nel tuo esigere\\
		nel tuo suggere\\
		nel tuo uccidere\\
		il saldo d’un debito
	\end{verse}

	\begin{verse}
		ci sia permesso\\
		di amare in te\\
		la tua crudeltà\\
		il tuo nascondere\\
		il tuo disvelare\\
		il tuo farti\\
		itinerario
	\end{verse}


\clearpage

\poemtitle{generazione di mostri}


	\begin{verse}
		per primi germinasti\\
		i mostri azzurri\\
		secreti di grembo estroso\\
		che era ignara ancora\\
		dei vomeri la terra\\
		anomica ancora\\
		molto prima che tutta intera\\
		si afformicolasse di uomini
	\end{verse}


\clearpage

\poemtitle{generazione di dei}


	\begin{verse}
		secondi generasti gli dei
	\end{verse}

	\begin{verse}
		parole non invocate\\
		parole vergini li facesti\\
		quiete incorrotta\\
		crudeltà ripiegata\\
		senza vittime ancora\\
		senza templi\\
		né tempo
	\end{verse}

	\begin{verse}
		ché ne inscrivesti le vite\\
		in troppo vasto circolo\\
		vite di cristalli che mutano\\
		lentamente
	\end{verse}

	\begin{verse}
		che non sanno l’andare
	\end{verse}


\clearpage

\poemtitle{generazione di uomini}


	\begin{verse}
		ci ripensasti
	\end{verse}

	\begin{verse}
		li avevi fatti ignari\\
		di colpa gli dei\\
		quasi fuori del nomos
	\end{verse}

	\begin{verse}
		e così germinasti\\
		gli uomini\\
		né spuma né abisso\\
		né aureo medio\\
		tra mostro e dio\\
		ma in bilico\\
		non stranieri alle vette\\
		al tartaro neppure stranieri
	\end{verse}

	\begin{verse}
		ma disposti ad andare\\
		i fianchi serrati\\
		i sandali ai piedi\\
		non ignari di urgenza
	\end{verse}


\clearpage

\poemtitle{scoperta della terra}


	\begin{verse}
		solchi di ombra nella terra\\
		solchi nella carne gravida\\
		sole estrae ombre gialle\\
		vapori sospiri di bestia
	\end{verse}

	\begin{verse}
		la sera ingiungono partenza\\
		i vapori che cercano il cielo\\
		nati siamo stranieri alla terra\\
		messe di tutte le terre
	\end{verse}

	\begin{verse}
		e ancora sapore di morsi\\
		buona carne del mio sangue\\
		e dei miei morsi nella carne\\
		uguale me al sole evocante vapori
	\end{verse}


\clearpage

\poemtitle{nomos}


	\begin{verse}
		sovrano Nomos\\
		amplesso eterno di tutti gli esseri\\
		di ognuno egualmente governi il respiro vitale\\
		a nessuno risparmiando la cognizione\\
		di una soggezione dappertutto eguale\\
		ma più dura ai mortali
	\end{verse}


\clearpage

\poemtitle{mare ii}


	\begin{verse}
		fatto marea\\
		di lava cobalto\\
		monti pian piano\\
		come sonno che stemperi\\
		il confine della coscienza
	\end{verse}

	\begin{verse}
		ti concedi\\
		cautamente\\
		all’ambiguità\\
		o la ritrovi
	\end{verse}

	\begin{verse}
		e sei cavità mite\\
		e desiderio erto\\
		ventre senza confini\\
		e impeto di voglia
	\end{verse}

	\begin{verse}
		pronto a poco a poco\\
		al tuo desiderio\\
		ad accoglierti\\
		a conoscerti\\
		a te stesso
	\end{verse}


\clearpage

\poemtitle{cicladi}


	\begin{verse}
		asciugandoti\\
		t’irrigidisci\\
		contraendoti\\
		ti screpoli\\
		in mille grani\\
		esplodi\\
		in schegge\\
		di granata\\
		rapprese\\
		in marmo alieno\\
		troppo bianco\\
		in aghi duri di sole\\
		tra stecche socchiuse
	\end{verse}

	\begin{verse}
		ti si levigano\\
		le coste ignote\\
		in liscia vigilanza\\
		in candido logos\\
		in autocoscienza\\
		frammenti di te\\
		che lasci andare\\
		subito meno tuoi\\
		benché\\
		a dirla tutta ti ci\\
		nuotano sopra come\\
		fave nel brodo
	\end{verse}


\clearpage

\poemtitle{argonauti}


	\begin{verse}
		così nuovo il mondo\\
		ha già confine
	\end{verse}

	\begin{verse}
		o limite o termine\\
		estremo bordo\\
		recinto\\
		frontiera
	\end{verse}

	\begin{verse}
		misura
	\end{verse}

	\begin{verse}
		oltre cui non va\\
		neppure la lancia scagliata\\
		da un passo\\
		ma dicono resti confitta nell’aria
	\end{verse}

	\begin{verse}
		vado
	\end{verse}

	\begin{verse}
		lascio\\
		il mio campo\\
		raccolgo brigata scelta di compagni\\
		preziosa agli dei gioventù\\
		se possa servire a obbligarne i favori\\
		un dono di carne lungo la strada
	\end{verse}

	\begin{verse}
		poi tornare indietro a contare\\
		magari con meno abbondandante\\
		messe d’eroi\\
		magari lasciata cadere\\
		tributo all’andare\\
		mia pelle già stata\\
		già storia\\
		io stesso mutato
	\end{verse}

	\begin{verse}
		è mio il viaggio\\
		mio l’andare\\
		e cosa esigerà il mare\\
		in cambio di lasciarmi passare\\
		che io non possa dargli?
	\end{verse}

	\begin{verse}
		io voglio\\
		strofinare il ventre\\
		delle mie navi sul suo\\
		ventre antico
	\end{verse}

	\begin{verse}
		e poi non è vero che il mondo\\
		sia nuovo\\
		così tanto
	\end{verse}


\poemtitle{stasimo i}


	\begin{verse}
		due fili gemelli ci sono\\
		e uno è presso il confine\\
		i due s’intrecciano\\
		senza stridere
	\end{verse}

	\begin{verse}
		due rivoli d’acqua ci sono\\
		gemelli che cercano il mare\\
		e uno è presso il confine\\
		i due si mescolano\\
		nascostamente\\
		sotterranee acque
	\end{verse}

	\begin{verse}
		e ogni foglia giovinetta sui rami\\
		del mio campo\\
		ha l’eguale presso il confine\\
		e così ogni petalo e spine\\
		di rose selvatiche
	\end{verse}

	\begin{verse}
		di ogni specie vivente\\
		pesci e uccelli e serpenti tra l’erba\\
		di ogni specie due gemelli vivono\\
		uno qui accanto a me\\
		e l’altro presso il confine\\
		accordando entrambi la rapida vita\\
		su un modo comune
	\end{verse}

	\begin{verse}
		e di profumi e di balsami dolci\\
		di spezie e d’unguenti d’Oriente\\
		che si danno l’oncia al prezzo\\
		di una notte\\
		e d’armonie di cimbali e d’àuli\\
		e di versi misurati dall’aedo\\
		e di veli di regine\\
		e di maschere di morti\\
		nulla esiste che non abbia
		presso il confine il suo gemello
	\end{verse}

	\begin{verse}
		luce di candele\\
		e le ombre\\
		guizzanti sui muri\\
		e calici di vino\\
		accanto a una caccia copiosa\\
		accanto al trofeo di un carniere rigonfio
	\end{verse}

	\begin{verse}
		e conchiglie e gabbiani e stelle marine\\
		gemelli da esibire in reti gemelle\\
		e alberi di navigli e sartie e fasciame\\
		perduti sul mare...
	\end{verse}

	\begin{verse}
		io stesso fui\\
		conchiglia delicata\\
		muto pesce nel mare\\
		pianta e uccello\\
		e fanciulla e ragazzo\\
		ma ora sento la mia vita\\
		ridiscendere la china\\
		inseguire le origini
	\end{verse}


\clearpage

\poemtitle{ellade}


	\begin{verse}
		cocci nell’arco luminoso del mattino\\
		caleidoscopio di sprazzi rifratti\\
		dispendio di seme splendente nella luce
	\end{verse}

	\begin{verse}
		poi frammenti di nero e di rosso\\
		dispersi nel mare
	\end{verse}

	\begin{verse}
		la terra\\
		quella pesante\\
		quella ferma\\
		ma strattonata dal mare\\
		più verso il basso\\
		la terra\\
		ha ogni pietra dipinta di sangue\\
		di roba\\
		splancnica\\
		di bile nera
	\end{verse}

	\begin{verse}
		ancora difficile o retrospettivamente\\
		millantato\\
		il sorgere del logos
	\end{verse}


\clearpage

\poemtitle{minotauro}


	\begin{verse}
		ogni andito\\
		ogni piega\\
		ogni diverticolo\\
		del labirinto\\
		promettono fuga\\
		ma insidiosi mi tengono\\
		tra le mura della tana\\
		levigate\\
		immutabili\\
		conosciute
	\end{verse}

	\begin{verse}
		ormai non conto più i giorni\\
		finché si incastrino in schema\\
		sempre eguale le costellazioni\\
		finché colmino il perimetro\\
		del mio groviglio\\
		gli stridi di fanciulli impuberi\\
		due volte sette\\
		femmine e maschi in parti eguali\\
		e quasi\\
		ignari di sesso
	\end{verse}

	\begin{verse}
		si aspettano che ne faccia scempio\\
		e mi fuggono addosso\\
		inondati da paura di bestie
	\end{verse}

	\begin{verse}
		sempre eguale terrore\\
		se ne colmano gli sguardi\\
		sempre eguale\\
		gli schiaffeggia le nari\\
		la fragranza della mia carne\\
		sempre eguale a essere\\
		troppo pronta a dargli retta
	\end{verse}

	\begin{verse}
		loro già edotti che il labirinto\\
		sa troppi ingressi e nessun esito\\
		che non sia la mia tana\\
		che non sia la mia carne\\
		né di costoro è concesso liberarmi
	\end{verse}

	\begin{verse}
		si aspettano che ne faccia scempio\\
		e mi fuggono addosso\\
		inondati da voglia di bestie\\
		e un lampo di sfida o trionfo\\
		negli sguardi e lo indovinano\\
		che magari gli invidio\\
		la rapida fine
	\end{verse}

	\begin{verse}
		la loro carne che già si apre\\
		pressata dalla mia sempre\\
		troppo prevedibilmente ridesta\\
		da terrore o voglia di bestia\\
		e intanto ho perso il conto\\
		delle volte e degli squarci
	\end{verse}

	\begin{verse}
		attendo soltanto\\
		chi il nomos comandi di spezzare\\
		la ciclicità di questo tempo\\
		perso nel groviglio di me\\
		attendo soltanto ormai\\
		un punto e basta
	\end{verse}

	\begin{verse}
		e poi il terrore a mia volta\\
		mi prende e mi chiedo\\
		se non sia già giunto\\
		se non sia perso anche lui\\
		nel suo labirinto\\
		se non sia più questo il tempo\\
		o il luogo giusto per attenderlo\\
		se non siano anche i suoi occhi\\
		dilatati da sensi bestiali\\
		o se non venga qui per irridermi\\
		giorno per giorno
	\end{verse}

	\begin{verse}
		lascio a marcire\\
		carogne di gioventù\\
		chiedendo che lo trascini\\
		fin quaggiù la voglia\\
		se ne abbia di eguali alle mie
	\end{verse}

	\begin{verse}
		e già si approssima la sizigia\\
		degli astri nel cielo\\
		e già ho deciso\\
		rifiuterò stavolta il pasto\\
		rifiuterò lo scempio\\
		che non sia forse\\
		tra i due volte sette giovani\\
		il liberatore
	\end{verse}

	\begin{verse}
		ma se così non sia\\
		risparmiandoli tutti\\
		li farò miei compagni nell’attesa\\
		riservando loro una sorte\\
		certo peggiore\\
		eguale alla mia
	\end{verse}


\clearpage

\poemtitle{mare iii}


	\begin{verse}
		rimpiattino di piaghe\\
		sparse di sabbia\\
		di sale di schegge\\
		di carapaci spezzati
	\end{verse}

	\begin{verse}
		intrecci di luci e cupezze\\
		tela franta dei fondali\\
		rubrica di coralli\\
		toccamenti di cnidarie
	\end{verse}

	\begin{verse}
		farandole di alghe\\
		astuzie di polpi\\
		matte cavalcate\\
		di cavalli di mare
	\end{verse}

	\begin{verse}
		quali mostri increati\\
		insidie mirabili\\
		pensi ancora nei gorghi\\
		quanti ne ascondi?
	\end{verse}

	\begin{verse}
		ma però quanta\\
		infinita voglia di esplorarti\\
		ci hai scritto tu dentro\\
		che ti cercassimo\\
		fino alle coste remote\\
		fino ai fondali del profondo?
	\end{verse}


\clearpage

\poemtitle{stasimo ii}


	\begin{verse}
		nessun gesto è più antico\\
		che impastare il pane\\
		meno che mettere a dormire\\
		le ossa dei morti\\
		meno che dar sollievo alla carne\\
		se chieda raschiamenti ulteriori
	\end{verse}

	\begin{verse}
		nessun gesto è più antico\\
		che affondare il coltello\\
		nella carne calda d’un uomo\\
		che sia nemico o compagno\\
		o giovane vergine dio\\
		da restituire ai suoi pari
	\end{verse}

	\begin{verse}
		si mangia poi\\
		per assorbire\\
		per comprendere\\
		perché non vada perduto midollo vitale\\
		né si sciolga continuità tra i mortali
	\end{verse}

	\begin{verse}
		sono lo stesso pietà e violenza
	\end{verse}


\clearpage

\poemtitle{limite}


	\begin{verse}
		terre\\
		protese nel mare\\
		manciate di sassi\\
		ambiguità di frontiere\\
		ogni momento riplasmate\\
		dal gioco dei flutti
	\end{verse}

	\begin{verse}
		come non lasciarsi\\
		crescere e nutrire\\
		dall’idea di avere di fronte\\
		dall’idea di tendere\\
		verso l’altro e l’altrove?
	\end{verse}

	\begin{verse}
		coscienza nascente\\
		di un’umanità\\
		che da ogni parte affolla le rive\\
		e incrocia cammini\\
		e tesse incontri\\
		sul mare
	\end{verse}


\clearpage

\poemtitle{mare iv}


	\begin{verse}
		maestà di tutte le cose\\
		che sono\\
		tu tutto del cosmo\\
		in te riposo\\
		signore luminoso
	\end{verse}

	\begin{verse}
		oltre le pluralità apparenti\\
		tu insegni che uno è ciò che è
	\end{verse}

	\begin{verse}
		sovrano di tutte le cose\\
		che scorrono\\
		origine di ogni movimento\\
		fulcro di ogni ciclo\\
		tu tutto produci\\
		a tutto dai incremento\\
		tutto accogli nel tuo grembo\\
		nel termine che il nomos decreta
	\end{verse}

	\begin{verse}
		su di te s’impernia la distesa\\
		della terra\\
		e l’uno è all’altra\\
		vicendevolmente\\
		complemento e sponda\\
		necessità e sostegno\\
		e scelta
	\end{verse}


\clearpage

\poemtitle{sirene}


	\begin{verse}
		seduzione\\
		sviamento\\
		naufragio
	\end{verse}

	\begin{verse}
		volle trovare in noi sguardo d’uomo
	\end{verse}

	\begin{verse}
		ma noi eravamo semplicemente\\
		attesa
	\end{verse}

	\begin{verse}
		ora mutarsi in schiuma\\
		non altra via che questa\\
		ché tanto\\
		tutto muore nel mare\\
		e rivive
	\end{verse}


\clearpage

\poemtitle{odisseo}


	\begin{verse}
		le lasciai da un pezzo le Sirene\\
		sono ancora là temo\\
		prigioniere di un prato fiorito\\
		verrà anche per loro l’incontro\\
		e la consapevolezza?
	\end{verse}

	\begin{verse}
		quanto a me\\
		non so se confessare che\\
		ho nostalgia di quei giorni
	\end{verse}

	\begin{verse}
		ormai è parecchio che vado per isole\\
		che mi perdo nel cabotaggio piccolo
	\end{verse}

	\begin{verse}
		l’approdo lo tengo lontano\\
		ci son nato sì\\
		ci ho conosciuto donna\\
		ho generato
	\end{verse}

	\begin{verse}
		ma è troppo grande il mondo là fuori\\
		troppa febbre brucia ancora di vita\\
		e Itaca è l’approdo e la pace
	\end{verse}

	\begin{verse}
		ho imparato ad apprezzare quel che si dà\\
		di umano tra due che s’incontrano per caso\\
		in un mercato di terre lontane
	\end{verse}

	\begin{verse}
		lo sguardo d’intesa\\
		una stretta di mano virile\\
		il calore dell’accoglienza\\
		e anche la furbizia e il raggiro\\
		tutto quanto c’è di umano negli uomini
	\end{verse}

	\begin{verse}
		come esser soli\\
		se ogni uomo\\
		ogni straniero\\
		lo saluti compagno\\
		alla smania che ti agita il sangue\\
		se senti tua ogni terra\\
		affacciata su un mare comune?
	\end{verse}


\clearpage

\poemtitle{calipso}


	\begin{verse}
                agli antichi tempi\\
                appartiene il tuo amore\\
                quando altri dèi altro sole\\
                presiedevano al destino
	\end{verse}

	\begin{verse}
                però qui il sonno\\
                di volta a volta\\
                farmaco prezioso\\
                ladro di sperienze\\
                scambia con il dolore oblio\\
                con vita altra vita
	\end{verse}

	\begin{verse}
                vorrei che morisse il tuo amore\\
                e poi nel giorno nuovo\\
                rinascesse intatto\\
                come un granaio sigillato
	\end{verse}

	\begin{verse}
                però qui il tempo\\
                s’avvolge qualche volta\\
                in riccioli o spire\\
                e sottrae agli sguardi\\
                il suo gioco segreto
	\end{verse}

	\begin{verse}
                elisir di sole di terra d’erbe\\
                mia patria trasmutata\\
                che a me ti nascondevi\\
                a dodici passi appena\\
                t’ho ritrovata\\
                son vicino a perderti ancora
	\end{verse}

	\begin{verse}
                però io disperso\\
                in un letto a difendermi da tedio\\
                o peggio da carezze\\
                solventi sleali\\
                segno della tua cura\\
                o a ritrovarmi in autocratici gesti
	\end{verse}

	\begin{verse}
                mille volte l’ho passata\\
                stanotte la frontiera\\
                misticanza di sonno e veglia\\
                esplosione di frammenti
	\end{verse}

	\begin{verse}
                quali ragioni dentro questa smania\\
                di cercare sempre nel presente\\
                un andersh una fuga\\
                un essere diverso\\
                il permesso di vivere portandomi\\
                un segno segreto sottopelle?
	\end{verse}


\clearpage

\poemtitle{eroe}


	\begin{verse}
		immergere mani\\
		in polle grosse\\
		delle profondità\\
		in quelle gravi\\
		di sanie
	\end{verse}

	\begin{verse}
		urgere ansiti\\
		da escrescenze\\
		livide\\
		astiose\\
		di polpo stanato\\
		estrarne stille
	\end{verse}

	\begin{verse}
		sempre prima snudarsi\\
		ma ditemi dove resta l’eroe\\
		lasciato a mezzarsi\\
		a estri altrui?
	\end{verse}

	\begin{verse}
		o magari insegnare a queste serpi\\
		di dita maestria di studi\\
		orgoglio di esecuzione\\
		autogestita
	\end{verse}


\clearpage

\poemtitle{stasimo iii}


	\begin{verse}
                dei giorni ci siano preziose\\
                anche queste cose\\
                lo sguardo duro dell’avversario\\
                la gemma amara dell’ingiustizia\\
                la parola che scopre un compagno nell’uomo\\
                che ci passa accanto
	\end{verse}


\clearpage

\poemtitle{padre o lo straniero}


	\begin{verse}
		da non lasciarsi conoscere\\
		se non per via di violenza\\
		se non da chi ignora\\
		che sia vergogna\\
		che sia timore\\
		che sia angoscia\\
		che sia censura\\
		la sua carne
	\end{verse}

	\begin{verse}
		troppo coriacea\\
		troppo antica\\
		troppo sedimentata\\
		la geologia di quella virilità\\
		perché ne sia questione lieve
	\end{verse}

	\begin{verse}
		ma che sia\\
		carne di uomo la sua carne\\
		e non granito senza speranza\\
		lo seppi già varcata la frontiera\\
		con Itaca alle spalle\\
		all’arco delle sue spalle\\
		ritrose quando non guardi\\
		e tenere e lo seppi\\
		alle sue mani e lo seppi ancora\\
		alla sua carne da dischiudere\\
		per uscirne alla luce
	\end{verse}


\clearpage

\poemtitle{terra}


	\begin{verse}
		carne di tutti costoro\\
		che lontani da Itaca\\
		o risparmiati\\
		o rectius sputati dal mare\\
		si abbracciano alle rene\\
		di questa terra
	\end{verse}

	\begin{verse}
		carne salda e vita sapiente\\
		schermaglie virili di mani di sguardi\\
		scambiati per strada\\
		parole di succo appena offuscate\\
		dal crescere lentissimo dei giorni\\
		pienezza di non essere io uno soltanto\\
		ma parte di molti\\
		molti io stesso
	\end{verse}


\clearpage

\poemtitle{mare v}


	\begin{verse}
		dall’alto vorrei saperti\\
		come albatro o gabbiano\\
		che abita le tempeste\\
		che non teme canto di risacca
	\end{verse}

	\begin{verse}
		dall’alto vorrei tracciarti\\
		i confini infinitamente intarsiati\\
		le tue curve frattali\\
		più dolci delle ànche dell’amato
	\end{verse}

	\begin{verse}
		sono grani d’uva Tiro\\
		ed Efeso e Smirne\\
		e Cuma e Alessandria\\
		grani d’uva di Corinto\\
		afformicolati di uomini
	\end{verse}

	\begin{verse}
		sono cascate di guizzi\\
		sul crespo dell’acqua\\
		fremiti di corde di cetra\\
		e campi di croquet\\
		di salti di delfini
	\end{verse}

	\begin{verse}
		sono amabili\\
		anche i mostri del tuo es\\
		anche le collere\\
		che ti corrono nel ventre
	\end{verse}

	\begin{verse}
		e dall’alto vorrei vederti\\
		e parlare di te ai compagni\\
		dall’alto magari d’un paio\\
		d’ali rubate ai gabbiani\\
		o artatamente divisate\\
		da un padre\\
		mastro d’inciarmi
	\end{verse}


\clearpage

\poemtitle{stasimo iv}


	\begin{verse}
		non è difficile sopravvivere\\
		del passato è sufficiente\\
		scordare fallimenti delusioni\\
		ed errori e sofferenze\\
		occasioni perdute e gli amori\\
		sciupati e i piaceri non saputi
	\end{verse}

	\begin{verse}
		non è difficile sopravvivere\\
		del domani non ammettere paure\\
		esibirsi indifferenti se all’orizzonte\\
		s’approssima il fantasma della vecchiaia\\
		come sera copre lungamente\\
		i passi dei monti
	\end{verse}

	\begin{verse}
		non è difficile sopravvivere\\
		se un altro giorno si sa\\
		s’aggiunge agli altri passati\\
                e dunque ignoriamo la fame\\
                e il sonno e la fatica\\
                straziamoci le carni\\
                affinché imparino a non concedersi\\
                al desiderio
	\end{verse}

	\begin{verse}
                non ci è permesso ricordare\\
                che è umana stirpe la nostra\\
                debole vulnerabile\\
                ma suscitiamo a noi stessi\\
                una corazza d'acciaio\\
                impenetrabile più che scudo d'Achille
	\end{verse}

\clearpage

\poemtitle{approdi?}

	\begin{verse}
		eppure mare c’era scritto\\
		nella carne dell’uomo\\
		non polvere\\
		né terra impastata a sudore\\
		ma acqua\\
		acqua salsa\\
		acqua di mare
	\end{verse}

	\begin{verse}
		e allora non ci è permesso ricordare\\
		che è umana stirpe la nostra\\
		umane vulnerabile stirpe
	\end{verse}

	\begin{verse}
		e ignoriamo la fame\\
		e sonno e fatica\\
		straziamoci le carni\\
		affinché imparino a non concedersi\\
		alla voglia di stare
	\end{verse}

	\begin{verse}
		nulla importa il gravame dei giorni\\
		pur di sfiorare le rive e andare\\
		non prima però di aver visto\\
		empori fenici e barattate\\
		ambre e conchiglie e saputi\\
		di prima mano i modi\\
		di amarsi nei porti sparsi\\
		per le rive ma poi andare
	\end{verse}

	\begin{verse}
		non per tornarne cospicui\\
		da pletorici bovi ammansiti\\
		come prescrivono i saggi\\
		a Itaca di pecunia o esperienza\\
		o dottrina o perché ne sia\\
		meglio pasciuta l’età grave\\
		sazia nel letto stato in caldo
	\end{verse}

	\begin{verse}
		ma perché tracciare solchi\\
		con le rapide navi sull’acqua\\
		perché andare oltre il giro\\
		stretto della terra\\
		oltre la solidità\\
		cara sotto i piedi\\
		è destino d’uomo
	\end{verse}

	\begin{verse}
		né conta che sia di piccolo\\
		cabotaggio la rotta di Odisseo\\
		o di mare alto
	\end{verse}

	\begin{verse}
		perché andare ancora e ancora\\
		e sì dispiegare ovunque astuzie\\
		e voglia e tracotanza e male\\
		a star lontani è umano\\
		ma prestare orecchio\\
		al canto di un altrove\\
		onnipresente\\
		è destino d’uomo
	\end{verse}

	\begin{verse}
		perché andare per andare\\
		è destino di quanti ci sono mortali\\
		e lasciarne i più fortunati\\
		al mare gravame di carne\\
		quando nomos fa essere sera
	\end{verse}

	\begin{verse}
		sacertà di quanto eternamente muta\\
		eternamente a sé eguale\\
		che non sanno misurare né il tempo\\
		stabilito dalla madre\\
		né l’esasperante incespicare dei secoli\\
		né l’avvicendarsi piatto di stagioni\\
		di opere e di giorni
	\end{verse}

	\begin{verse}
		era scritto mare in noi\\
		e doveva essere\\
		mare
	\end{verse}
