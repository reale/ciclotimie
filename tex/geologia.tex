\chapter{Geologia}

\poemtitle{i}

\begin{verse}
    a Meleagro\\
    il figlio d'Oineo\\
    la madre gli mise la vita nel fuoco\\
    bastò un gesto della mano
\end{verse}

\begin{verse}
    del padre invece non raccontano molto
\end{verse}

\clearpage

\poemtitle{ii}

\begin{verse}
    tutti quanti incontriamo dopo —\\
    nessuno di loro ci procura la fine\\
    ché sono in genere soltanto compagni\\
    o alla peggio lupi maldestri
\end{verse}

\begin{verse}
    e non danno\\
    — aspra talvolta —\\
    che vita
\end{verse}

\begin{verse}
    invece il modo e il tempo della fine\\
    segnati già nel modo di toccare\\
    o nel primitivo sguardo\\
    o nel negarsi saldo di questo\\
    li decide la madre
\end{verse}

\clearpage

\poemtitle{iii}

\begin{verse}
    gli occhi dove non fu mai remissione\\
    non so se li nascondesti da noi\\
    o soltanto li negavi al mondo\\
    ma noi noi li cercavamo da prima ancora\\
    che ci gettasti sulle spiagge di luce\\
    né ancora abbiamo smesso
\end{verse}

\clearpage

\poemtitle{iv}

\begin{verse}
    mostro i denti arretrando\\
    a un approccio di madre\\
    a una carezza tardiva\\
    come ad agguato di belva\\
    gelosa del sangue
\end{verse}

\begin{verse}
    non sono che scolta ignara\\
    di sosta denti serrati\\
    allarme di lupo stretto da cani\\
    sguardo fisso in sguardo\\
    primario impervio di madre
\end{verse}

\clearpage

\poemtitle{v}

\begin{verse}
    della terra\\
    è impura — mi dissero —\\
    tu non mescolartici\\
    e fu tra le beffe\\
    non la meno atroce
\end{verse}

\begin{verse}
    della terra\\
    solo poi mi feci esperto\\
    la sanno il mio dorso il petto\\
    le braccia che la tengono ferma
\end{verse}

\begin{verse}
    della terra\\
    so cosa amo\\
    non l'accoglienza di argille di sabbia\\
    ma il granito aspro\\
    senza cedimenti\\
    indifferente
\end{verse}

\clearpage

\poemtitle{vi}

\begin{verse}
    né macigno da scagliare\\
    per detronizzarti\\
    né per farmene\\
    difesa da te\\
    estrema e fragile
\end{verse}

\begin{verse}
    era per sederci\\
    insieme e per parlarci\\
    e per vederci\\
    come avremmo dovuto
\end{verse}

\begin{verse}
    non scoprirti a scegliermi\\
    il peggiore tra i nemici\\
    il più vile\\
    non per sapermi\\
    capace di violarti\\
    che neppure ti arrivavo\\
    con la testa al petto
\end{verse}

\clearpage

\poemtitle{vii}

\begin{verse}
    nel momento di far quadra dei conti\\
    so che non si colpisce per lacerare\\
    ma per scoprire la carne\\
    per sapere la geologia dei legami
\end{verse}

\begin{verse}
    che non siamo di sostanza nemica\\
    lo scopersi nelle tue spalle\\
    lo riconobbi al gesto con cui ti schermivi\\
    e ci fa vicini vergogna e ribellione\\
    e dispersa sapienza d’amore
\end{verse}

\begin{verse}
    ora nelle tue mani non trovo più\\
    angoscia — di quella non voglio parlare —\\
    ma voto d’intelletto reciproco\\
    e lo so il segreto — che nascondi\\
    così bene — ma t'ho visto —
\end{verse}

\begin{verse}
    voglio che sia scelta\\
    il nostro primo incontro
\end{verse}

\clearpage

\poemtitle{viii}

\begin{verse}
    da che li seppi ignari\\
    di umanità corrente\\
    li volli incolpevoli automi\\
    né ho poi più lasciato\\
    di cercarli altrove
\end{verse}
